\documentclass[12pt, leqno, british]{amsart}
\usepackage[style=alphabetic, backend=biber]{biblatex}
\usepackage{a4, amsmath}
\usepackage{mathtools}
\usepackage{amssymb}
\usepackage{amsthm, amscd, mathdots}
\swapnumbers
\usepackage{enumerate}
\usepackage{hyperref}
\usepackage{cleveref}
\usepackage{csquotes}
\usepackage{color}
\usepackage{datetime}
\usepackage{xr, standalone, import}

\theoremstyle{definition}
\newtheorem{defi}{Definition}[subsection]
\theoremstyle{plain}
\newtheorem{prop}[defi]{Proposition}
\newtheorem{lem}[defi]{Lemma}
\newtheorem{thm}[defi]{Theorem}
\newtheorem{cor}[defi]{Corollary}
\newtheorem{ques}[defi]{Question}
\theoremstyle{remark}
\newtheorem{rem}[defi]{Remark}
\newtheorem{eg}[defi]{Example}
\newtheorem{egs}[defi]{Examples}

\newcommand{\mc}{\mathcal}
\newcommand{\mf}{\mathfrak}
\newcommand{\mbb}{\mathbb}
\newcommand{\nat}{\mbb N}
\newcommand{\cc}{\mathbb C}
\newcommand{\rr}{\mathbb R}
\newcommand{\qq}{\mbb Q}
\newcommand{\ovl}{\overline}
\newcommand{\ff}{\mbb F}
\newcommand{\zz}{\mbb Z}

\DeclareMathOperator{\charac}{char}
\DeclareMathOperator{\id}{id}
\DeclareMathOperator{\Frac}{Frac}
\DeclareMathOperator{\Ker}{Ker}
\DeclareMathOperator{\Img}{Im}
\DeclareMathOperator{\Trd}{Trd}
\DeclareMathOperator{\Tr}{Tr}
\DeclareMathOperator{\Nrd}{Nrd}
\DeclareMathOperator{\GL}{GL}
\DeclareMathOperator{\Gal}{Gal}
\DeclareMathOperator{\ord}{ord}
\DeclareMathOperator{\trdeg}{trdeg}
\DeclareMathOperator{\supp}{supp}
\DeclareMathOperator{\rad}{rad}
\DeclareMathOperator{\sign}{sign}
\newcommand{\disc}{\mathrm{d}}

\newcommand{\llangle}{\langle\!\langle}
\newcommand{\rrangle}{\rangle\!\rangle}
\addbibresource{bibliography.bib}
\externaldocument[M-]{./Lecture-notes}

\title{Quadratic forms and class fields II: lecture notes}
\author{Nicolas Daans}
\date{\today}
\address{Charles University, Faculty of Mathematics and Physics, Department of Algebra, Sokolov\-sk\' a 83, 18600 Praha~8, Czech Republic.}
\email{nicolas.daans@matfyz.cuni.cz}

\makeindex
\begin{document}
\maketitle
\tableofcontents

\subsection*{Notations and conventions}
We denote by $\nat$ the set of natural numbers.
We write $\nat^+$ for the proper subset of non-zero numbers.
For a ring $R$, we denote by $R^\times$ the set of invertible elements of $R$; if $R$ is a field, then $R^\times = R \setminus \lbrace 0 \rbrace$.
We will denote the set of prime numbers by $\mathbb{P}$.

\subsection*{Acknowledgements}
The first part of the course (lectures 1--5) follows to a large extent the exposition from Lam's book \autocite{Lam}.
For this introductory course, we focus on fields of characteristic different from $2$, where the theory of quadratic forms is simpler than over fields of characteristic $2$.
The book of Elman, Karpenko and Merkurjev \autocite{ElmanKarpenkoMerkurjev} is a great reference for those who want to learn more about quadratic form theory over fields of arbitrary characteristic, and some parts of this course which hold in arbitrary characteristic, are inspired by their work.
I thank Ruben de Preter for helpful feedback on the previous year's lecture notes.
Finally, I gratefully acknowledge the inspiration taken from the course ``Quadratic Forms'' taught by Karim Johannes Becher at the University of Antwerp in Belgium, which has to a large extent shaped my vision on modern quadratic form theory.

The second part of the course (lectures 6--10) was previously taught by B{\l}a{\.z}ej {\.Z}mija, and I have taken inspiration from his classes and lecture notes.
The proof of the Hasse-Minkowski Theorem in particular is inspired in turn by a note of Hatley \cite{Hat}.

\subimport{Lecture1/}{Lecture1}
\subimport{Lecture2/}{Lecture2}
\subimport{Lecture3/}{Lecture3}
\subimport{Lecture4/}{Lecture4}
\subimport{Lecture5/}{Lecture5}
\subimport{Lecture6/}{Lecture6}
\subimport{Lecture7/}{Lecture7}
\subimport{Lecture8/}{Lecture8}
\subimport{Lecture9/}{Lecture9}
\subimport{Lecture10/}{Lecture10}

\printindex
\printbibliography
\end{document}