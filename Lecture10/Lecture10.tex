\documentclass[12pt, leqno, british]{amsart}
\usepackage[style=alphabetic, backend=biber]{biblatex}
\usepackage{a4, amsmath}
\usepackage{mathtools}
\usepackage{amssymb}
\usepackage{amsthm, amscd, mathdots}
\swapnumbers
\usepackage{enumerate}
\usepackage{hyperref}
\usepackage{cleveref}
\usepackage{csquotes}
\usepackage{color}
\usepackage{datetime}
\usepackage{xr, standalone, import}

\theoremstyle{definition}
\newtheorem{defi}{Definition}[subsection]
\theoremstyle{plain}
\newtheorem{prop}[defi]{Proposition}
\newtheorem{lem}[defi]{Lemma}
\newtheorem{thm}[defi]{Theorem}
\newtheorem{cor}[defi]{Corollary}
\newtheorem{ques}[defi]{Question}
\theoremstyle{remark}
\newtheorem{rem}[defi]{Remark}
\newtheorem{eg}[defi]{Example}
\newtheorem{egs}[defi]{Examples}

\newcommand{\mc}{\mathcal}
\newcommand{\mf}{\mathfrak}
\newcommand{\mbb}{\mathbb}
\newcommand{\nat}{\mbb N}
\newcommand{\cc}{\mathbb C}
\newcommand{\rr}{\mathbb R}
\newcommand{\qq}{\mbb Q}
\newcommand{\ovl}{\overline}
\newcommand{\ff}{\mbb F}
\newcommand{\zz}{\mbb Z}

\DeclareMathOperator{\charac}{char}
\DeclareMathOperator{\id}{id}
\DeclareMathOperator{\Frac}{Frac}
\DeclareMathOperator{\Ker}{Ker}
\DeclareMathOperator{\Img}{Im}
\DeclareMathOperator{\Trd}{Trd}
\DeclareMathOperator{\Tr}{Tr}
\DeclareMathOperator{\Nrd}{Nrd}
\DeclareMathOperator{\GL}{GL}
\DeclareMathOperator{\Gal}{Gal}
\DeclareMathOperator{\ord}{ord}
\DeclareMathOperator{\trdeg}{trdeg}
\DeclareMathOperator{\supp}{supp}
\DeclareMathOperator{\rad}{rad}
\DeclareMathOperator{\sign}{sign}
\newcommand{\disc}{\mathrm{d}}

\newcommand{\llangle}{\langle\!\langle}
\newcommand{\rrangle}{\rangle\!\rangle}
\addbibresource{../bibliography.bib}
\externaldocument[M-]{../Lecture-notes}

\author{Nicolas Daans}
\address{Charles University, Faculty of Mathematics and Physics, Department of Algebra, Sokolov\-sk\' a 83, 18600 Praha~8, Czech Republic.}
\email{nicolas.daans@matfyz.cuni.cz}

\begin{document}

\section{Lecture 10}

\subsection{The Hasse-Minkowski Theorem}
We are ready to phrase and proof the Hasse-Minkowski Theorem.
\begin{thm}[Hasse-Minkowski]\label{T:Hasse-Minkowski}
Let $q$ be an anisotropic quadratic form over $\qq$.
Then there exists $p \in \mbb P'$ such that $q_{\qq_p}$ is anisotropic.
\end{thm}
In other words, if one wants to check whether a quadratic form over $q$ is isotropic, if suffices to prove that it is isotropic over $\qq_p$ for all $p \in \mbb P'$.
This might a priori seem like an equally hard problem since one needs to consider infinitely many prime numbers $p \in \mbb P'$, but recall that if $\dim(q) \geq 3$, then $q$ contains a subform isometric to $\langle a_1, a_2, a_3 \rangle_\qq$ for some $a_1, a_2, a_3 \in \zz \setminus \lbrace 0 \rbrace$, and then $q_{\qq_p}$ is automatically isotropic for all $p \in \mbb P \setminus \lbrace 2 \rbrace$ which do not divide $a_1$, $a_2$ and $a_3$ by \Cref{M-L:qqp-odd-3d-form}.
This leaves only finitely many prime numbers to check, and checking isotropy of quadratic forms over $\qq_p$ can be done algorithmically, as explained in lectures $7$ and $8$.

\Cref{T:Hasse-Minkowski} has the following immediate consequence, showing that we obtain a complete classifiction of quadratic forms over $\qq$ via the classification of quadratic forms over each $\qq_p$.
\begin{cor}\label{C:Hasse-Minkowski}
Let $q_1, q_2$ be quadratic forms over $\qq$.
If $(q_1)_{\qq_p} \cong (q_2)_{\qq_p}$ for all $p \in \mbb P'$, then $q_1 \cong q_2$.
In particular, the group homomorphism
$$ W\qq \to \prod_{p \in \mbb P'} W\qq_p : [q] \mapsto ([q_{\qq_p}])_{p \in \mbb P'}$$
is injective.
\end{cor}
\begin{proof}
Exercise.
\end{proof}
\begin{proof}[Proof of \Cref{T:Hasse-Minkowski}.]
Let $n = \dim(q)$.
By \Cref{M-C:diagonalisation} we may assume without loss of generality $q = \langle a_1, \ldots, a_n \rangle_\qq$ for some $a_1, \ldots, a_n \in \qq$, and in fact we have $a_1, \ldots, a_n \in \qq^\times$ if we assume that $q$ is anisotropic.
By replacing $q$ with a similar quadratic form (which does not affect isotropy over $\qq$ or any field extension) we may assume that $a_1, \ldots, a_n$ are coprime, square-free integers.

We now consider several special cases.
For $n=1$ there is nothing to prove: $\langle a_1 \rangle_{\qq_p}$ is anisotropic for all $p \in \mbb P'$.

\emph{Assume $n=2$}.
By Exercise \eqref{M-ex:2-dim-form-isometric} of Lecture 1 we have that $-a_1a_2 \not\in \qq^{\times 2}$.
This implies that either $a_1a_2 > 0$, in which case $-a_1a_2 \not\in \rr^{\times 2}$ and hence $q_\rr$ is anisotropic, or there is $p \in \mbb P$ which divides $-a_1a_2$ an odd number of times, whereby $-a_1a_2 \not\in \qq_p^{\times 2}$ and hence $q_{\qq_p}$ is anisotropic.
This concludes the proof for $n = 2$.

\emph{Assume $n = 3$.}
We it suffices to consider the case where $a_3 = 1$ (see Exercise \eqref{ex:3-dim-reduction} below), and we further assume that $a_1$ and $a_2$ are square-free.
We assume that $q = \langle 1, a_1, a_2 \rangle_{\qq}$ is isotropic over $\qq_p$ for all $p \in \mbb P'$ and we need to show that $q$ itself is isotropic.
By switching the roles of $a_1$ and $a_2$, we may assume $\lvert a_1 \rvert \leq \lvert a_2 \rvert$.
In view of \Cref{M-C:representation-theorem} we equivalently need to show that $\langle 1, a_1 \rangle_{\qq}$ represents $-a_2$, assuming it does so over $\qq_p$ for all $p \in \mbb P'$.

We proceed by induction on $\lvert a_2 \rvert$.
If $\lvert a_2 \rvert = \lvert a_1 \rvert = 1$, then either $a_1 = -1$ or $a_2 = -1$, in which case $q$ is isotropic, or $a_1 = a_2 = 1$, but in this case $q_\rr = \langle 1, 1, 1 \rangle_\rr$ is anisotropic, contradicting the assumption.
Now assume $\lvert a_2 \rvert > 1$.
For each prime number $p$ dividing $a_2$ there exists $c \in \zz$ with $c^2 \equiv -a_1 \bmod p$: if $p$ divides $a_1$ one may just take $c = 0$, if $p = 2$ and $p$ does not divide $a_1$ one may take $c = 1$, and otherwise this follows from the fact that $q_{\qq_p}$ is isotropic and \Cref{M-L:qqp-odd-3d-form}.
Since $a_2$ is square-free, by the Chinese Remainder Theorem (\Cref{M-T:CRT}) we find $c \in \zz$ such that $c^2 \equiv -a_1 \bmod a_2$.
Clearly we may even find such $c$ with $\lvert c \rvert \leq \lvert a_2 \rvert/2$.

It follows that $c^2 = -a_1 + a_2b$ for some $b \in \zz$, whereby $a_2b \in D_\qq(\langle 1, a_1 \rangle_\qq)$.
Furthermore, we compute that
$$ \lvert b \rvert = \left\lvert \frac{c^2 - a_1}{a_2} \right\rvert \leq \left\lvert \frac{c^2}{a_2} \right\rvert + \left\lvert \frac{a_1}{a_2} \right\rvert \leq \frac{\lvert a_2 \rvert}{4} + 1 < \lvert a_2 \rvert.$$
If we had $b \not\in D_\qq(\langle 1, a_1 \rangle_{\qq})$, then by the induction hypothesis there exists $p \in \mbb P'$ such that $b \not\in D_{\qq_p}(\langle 1, a_1 \rangle_{\qq_p})$.
But since $\langle 1, a_1 \rangle_{\qq_p}$ is multiplicative and $a_2b \in D_{\qq_p}(\langle 1, a_1 \rangle_{\qq_p})$, it would follow that $a_2 \not\in  D_{\qq_p}(\langle 1, a_1 \rangle_{\qq_p})$, contradicting the assumption that $q = \langle 1, a_1, a_2 \rangle_{\qq}$ is isotropic over $\qq_p$ for all $p \in \mbb P'$.
Hence we must have $b \in D_\qq(\langle 1, a_1 \rangle_{\qq})$, and then by the multiplicativity of $ D_\qq(\langle 1, a_1 \rangle_{\qq})$ also $a_2 = (a_2b)/b \in D_{\qq}(\langle 1, a_1 \rangle_{\qq})$, hence we are done.
This concludes the proof for the case $n = 3$.

\emph{Assume $n = 4$}.
We assume that $q = \langle a_1, a_2, a_3, a_4 \rangle_{\qq}$ is isotropic over $\qq_p$ for all $p \in \mbb P'$ and show that $q$ is itself isotropic.

Let $T = \lbrace 2, \infty \rbrace \cup \lbrace p \in \mbb P \mid p\mid a_1a_2a_3a_4 \rbrace$.
This is a finite set.
By assumption $\langle a_1, a_2 \rangle_{\qq_p}$ and $-\langle a_3, a_4 \rangle_{\qq_p}$ represent a common element in $\qq_p$ for each $p \in \mbb P$.
In view of the classification of square classes (\Cref{M-C:qqp-squares-odd} and \Cref{M-C:qq2-squares}) we may assume there exists for each $p \in \mbb P'$ an element $z_p \in \zz \cap D_{\qq_p}(\langle a_1, a_2 \rangle_{\qq_p}) \cap D_{\qq_p}(-\langle a_3, a_4 \rangle_{\qq_p})$ with, for $p \in \mbb P$, $v_p(z_p) \leq 1$.
By \Cref{M-T:Dirichlet} we may find a prime $q \in \mbb P \setminus T$ and $\varepsilon \in \lbrace 1, -1 \rbrace$ such that, for $z = \varepsilon q$, $z z_\infty > 0$, $z \equiv z_2 \bmod 16$ and $z \equiv z_p \bmod p^2$ for $p \in T \setminus \lbrace 2, \infty \rbrace$.
It follows by \Cref{M-LocalSquareThm} that $z z_p \in \qq_p^{\times 2}$ for all $p \in T$, whence $z \in  D_{\qq_p}(\langle a_1, a_2 \rangle_{\qq_p}) \cap D_{\qq_p}(-\langle a_3, a_4 \rangle_{\qq_p})$ and thus $\langle a_1, a_2, -z \rangle_{\qq_p}$ and $\langle z, -a_3, -a_4 \rangle_{\qq_p}$ are isotropic for all $p \in T$.
Furthermore, in view of \Cref{M-L:qqp-odd-3d-form}, $\langle a_1, a_2, -z \rangle_{\qq_p}$ and $\langle z, -a_3, -a_4 \rangle_{\qq_p}$ are also isotropic for all $p \in \mbb P \setminus (T \cup \lbrace q \rbrace)$.
Finally, by \Cref{M-T:HilbertReciprocity} we conclude that $\langle a_1, a_2, -z \rangle_{\qq_q}$ and $\langle z, -a_3, -a_4 \rangle_{\qq_q}$ must also be isotropic, and then by the case $n = 3$ we conclude that $\langle a_1, a_2, -z \rangle_{\qq}$ and $\langle z, -a_3, -a_4 \rangle_{\qq}$ are isotropic, and hence in particular $q$ is isotropic.
This concludes the proof for the case $n = 4$.

\emph{Assume $n \geq 5$.}
We assume that $q = \langle a_1, \ldots, a_n \rangle_{\qq}$ is isotropic over $\qq_p$ for all $p \in \mbb P'$ and show that $q$ itself is isotropic.
Consider $T = \lbrace 2, \infty \rbrace \cup \lbrace p \in \mbb P \mid a_3a_4\ldots a_n \rbrace$.
This is a finite set.
As in the case $n = 4$, we find for each $p \in \mbb P'$ an element $z_p \in \zz \cap D_{\qq_p}(\langle a_1, a_2 \rangle_{\qq_p}) \cap D_{\qq_p}(-\langle a_3, \ldots, a_n \rangle_{\qq_p})$.
Let us write $z_p = a_1x_p^2 + a_2y_p^2$ for some $x_p, y_p \in \qq_p$; we may further assume $x_p$, $y_p \neq 0$ (see e.g.~Exercise \eqref{M-ex:basis-isotropic-vectors} of Section 2).
By \Cref{M-T:WAT} we may find $x, y \in \qq$ such that
\begin{align*}
v_p(x - x_p) &> \max \lbrace v_p(x_p), v_p(4z_p(a_1x_p)^{-1}) \rbrace \text{ for all } p \in T \setminus \lbrace \infty \rbrace,\\
v_p(y - y_p) &> \max \lbrace v_p(y_p), v_p(4z_p(a_2y_p)^{-1}) \rbrace \text{ for all } p \in T \setminus \lbrace \infty \rbrace,\\
\lvert x - x_\infty \rvert &< \min \lbrace \lvert x_\infty \rvert, \lvert z_\infty (4a_1x_\infty)^{-1} \rvert \rbrace, \text{ and} \\
\lvert y - y_\infty \rvert &< \min \lbrace \lvert x_\infty \rvert, \lvert z_\infty (4a_2y_\infty)^{-1} \rvert \rbrace.
\end{align*}
Now set $z = a_1x^2 + a_2y^2$.
It is clear that $z \in D_\qq(\langle a_1, a_2 \rangle_\qq)$ by construction, and we shall show that $z \in z_p\qq_p^{\times 2}$ for all $p \in T$, from which it follows that $z \in D_{\qq_p}(-\langle a_3, \ldots, a_n \rangle_{\qq_p})$ for all $p \in T$ and thus $\langle z, -a_3, \ldots, -a_n \rangle_{\qq_p}$ is isotropic.
Since we automatically have that $\langle z, -a_3, \ldots, -a_n \rangle_{\qq_p}$ is isotropic for $p \in \mbb P \setminus T$ (since already $-\langle a_3, \ldots, a_n \rangle_{\qq_p}$ is isotropic by \Cref{M-L:qqp-odd-3d-form}), we obtain by the induction hypothesis that $\langle z, -a_3, \ldots, -a_n \rangle_{\qq}$ is isotropic.
We conclude that $q$ is isotropic, as desired.

It remains to show the claim that $z \in z_p\qq_p^{\times 2}$ for all $p \in T$.
First consider $p \neq \infty$.
Observe that $v_p(x + x_p) \geq v_p(x_p)$ and $v_p(y + y_p) \geq v_p(y_p)$.
We compute that
\begin{align*}
&v_p(z - z_p) \\
=\enspace &v_p(a_1x^2 + a_2y^2 - a_1x_p^2 - a_2y_p^2) \\
=\enspace &v_p(a_1(x - x_p)(x + x_p) + a_2(y  - y_p)(y + y_p)) \\
\geq\enspace &\min \lbrace v_p(a_1) + v_p(x - x_p) + v_p(x + x_p), v_p(a_2) + v_p(y - y_p) + v_p(y + y_p) \rbrace \\
>\enspace &v_p(4z_p)
\end{align*}
from which it follows that $z \in z_p\qq_p^{\times 2}$ by Exercise \eqref{M-ex:local-squares-open} of Section 7.
For $p = \infty$, we compute similarly that
\begin{align*}
\lvert z - z_\infty \rvert \leq \lvert a_1 \rvert \lvert x - x_\infty \rvert \lvert x + x_\infty \rvert + \lvert a_2 \rvert \lvert y - y_\infty \rvert \lvert y + y_\infty \rvert < \lvert z_\infty \rvert
\end{align*}
whereby $z$ and $z_\infty$ must have the same sign and thus $z \in z_\infty\rr^{\times 2}$.
\end{proof}
\subsection{Exercises}
\begin{enumerate}
\item Give a proof of \Cref{C:Hasse-Minkowski}.
\item\label{ex:3-dim-reduction} Let $K$ be a field with $\charac(K) \neq 2$, $a_1, a_2, a_3 \in K^\times$.
Show that the following are equivalent:
\begin{itemize}
\item $\langle a_1, a_2, a_3 \rangle_{K}$ is isotropic,
\item $\llangle -a_2a_3, -a_1a_3 \rrangle_K$ is isotropic,
\item $\langle 1, -a_2a_3, -a_1a_3 \rangle_K$ is isotropic.
\end{itemize}
\item Compute $\lvert I^3 \qq/I^4 \qq \rvert$.
\item Let $K$ be a field with $\charac(K) \neq 2$.
We say that two $2$-fold Pfister forms $q_1$ and $q_2$ over $K$ are \emph{linked} if there exists $a, b, c \in K^\times$ such that $q_1 \cong \llangle a, b \rrangle_K$ and $q_2 \cong \llangle a, c \rrangle_K$.
\begin{itemize}
\item Show that two $2$-fold Pfister forms $\llangle a, b \rrangle_K$ and $\llangle c, d \rrangle_K$ are linked if and only if the quadratic form $\langle a, b, -ab, -c, -d, cd \rangle_K$ is isotropic, if and only if there exist $e, f \in K^\times$ such that $\lbrace a, b \rbrace_K + \lbrace c, d \rbrace_K = \lbrace e, f \rbrace_K$ in $I^2K/I^3/K$.
\item Show that over $K = \qq$, any two $2$-fold Pfister forms are linked.
\item Is it true that, for any $n \in \nat$ and $2$-fold Pfister forms $q_1, \ldots, q_n$ over $\qq$, there exists $a, b_1, \ldots, b_n \in \qq^\times$ such that $q_i \cong \llangle a, b_i \rrangle_{\qq}$ for all $i$?
\end{itemize}
\end{enumerate}

\end{document}