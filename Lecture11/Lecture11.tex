\documentclass[12pt, leqno, british]{amsart}
\usepackage[style=alphabetic, backend=biber]{biblatex}
\usepackage{a4, amsmath}
\usepackage{mathtools}
\usepackage{amssymb}
\usepackage{amsthm, amscd, mathdots}
\swapnumbers
\usepackage{enumerate}
\usepackage{hyperref}
\usepackage{cleveref}
\usepackage{csquotes}
\usepackage{color}
\usepackage{datetime}
\usepackage{xr, standalone, import}

\theoremstyle{definition}
\newtheorem{defi}{Definition}[subsection]
\theoremstyle{plain}
\newtheorem{prop}[defi]{Proposition}
\newtheorem{lem}[defi]{Lemma}
\newtheorem{thm}[defi]{Theorem}
\newtheorem{cor}[defi]{Corollary}
\newtheorem{ques}[defi]{Question}
\theoremstyle{remark}
\newtheorem{rem}[defi]{Remark}
\newtheorem{eg}[defi]{Example}
\newtheorem{egs}[defi]{Examples}

\newcommand{\mc}{\mathcal}
\newcommand{\mf}{\mathfrak}
\newcommand{\mbb}{\mathbb}
\newcommand{\nat}{\mbb N}
\newcommand{\cc}{\mathbb C}
\newcommand{\rr}{\mathbb R}
\newcommand{\qq}{\mbb Q}
\newcommand{\ovl}{\overline}
\newcommand{\ff}{\mbb F}
\newcommand{\zz}{\mbb Z}

\DeclareMathOperator{\charac}{char}
\DeclareMathOperator{\id}{id}
\DeclareMathOperator{\Frac}{Frac}
\DeclareMathOperator{\Ker}{Ker}
\DeclareMathOperator{\Img}{Im}
\DeclareMathOperator{\Trd}{Trd}
\DeclareMathOperator{\Tr}{Tr}
\DeclareMathOperator{\Nrd}{Nrd}
\DeclareMathOperator{\GL}{GL}
\DeclareMathOperator{\Gal}{Gal}
\DeclareMathOperator{\ord}{ord}
\DeclareMathOperator{\trdeg}{trdeg}
\DeclareMathOperator{\supp}{supp}
\DeclareMathOperator{\rad}{rad}
\DeclareMathOperator{\sign}{sign}
\newcommand{\disc}{\mathrm{d}}

\newcommand{\llangle}{\langle\!\langle}
\newcommand{\rrangle}{\rangle\!\rangle}
\addbibresource{../bibliography.bib}
\externaldocument[M-]{../Lecture-notes}

\author{Nicolas Daans}
\address{Charles University, Faculty of Mathematics and Physics, Department of Algebra, Sokolov\-sk\' a 83, 18600 Praha~8, Czech Republic.}
\email{nicolas.daans@matfyz.cuni.cz}

\begin{document}

\section{Lecture 11}
In this final lecture, we give another application of quadratic form theory of a more algorithmic nature: Hermite's method of counting real zeros of a univariate polynomial.

Let always $K$ be a field.

\subsection{The trace of an algebra}
Recall from linear algebra that the \emph{trace of a matrix $A$}with entries in $K$, denoted by $\Tr(A)$, is the sum of its diagonal elements.
It satisfies the property $\Tr(AB) = \Tr(BA)$ for square matrices $A, B$ of the same size.

When $V$ is a finite-dimensional $K$-vector space, we can thus associate to every linear map $L : V \to V$ the \emph{trace of the linear map $L$}\index{trace!of a linear map} , denoted by $\Tr(V)$, as the trace of the matrix of $L$ with respect to any choice of basis of $V$.
Note that, if $M_{\mc{B}}$ and $M_{\mc{B}'}$ are the matrices of $L$ with respect to different bases $\mc{B}$ and $\mc{B}'$ of $V$, then we have $M_{\mc{B}} = CM_{\mc{B}'}C^{-1}$ for an invertible base change matrix $C$, and then $\Tr(M_{\mc{B}}) = \Tr(CM_{\mc{B}'}C^{-1}) = \Tr(CC^{-1}M_{\mc{B}'}) = \Tr(M_{\mc{B}'})$, so this definition does not depend on the choice of basis.

Recall that a \emph{$K$-algebra} is a ring $\mc{A}$ containing $K$ and such that $k\alpha = \alpha k$ for all $\alpha \in \mc{A}$ and $k \in K$.
This last property is satisfied in particular if $\mc{A}$ is a commutative ring, which will be the case for all algebras which we consider in the next part.
$\mc{A}$ is a vector space over $K$, and we say that $\mc{A}$ is a finite-dimensional $K$-algebra if it is finite-dimensional as a vector space over $K$, and write $\dim(\mc{A})$ for the dimension of $\mc{A}$ as a $K$-vector space.
\begin{defi}\label{D:trace}
Let $\mc{A}$ be a finite-dimensional $K$-algebra, $\alpha \in \mc{A}$.
We define the \emph{trace of $\alpha$}\index{trace!of an element of an algebra}, denoted by $\Tr_{\mc{A}}(\alpha)$, as the trace of the $K$-linear map
$$ l_\alpha : \mc{A} \to \mc{A} : \beta \mapsto \alpha \beta.$$
\end{defi}
\begin{prop}\label{P:trace-properties}
Let $\mc{A}$, $\mc{B}$, $\mc{C}$ be finite-dimensional $K$-algebras.
The trace satisfies the following properties:
\begin{enumerate}
\item The map $\Tr_{\mc{A}} : \mc{A} \to K$ is $K$-linear.
%\item\label{it:trace-symmbil} The map $\mc{A} \times \mc{A} \to K : (\alpha, \beta) \mapsto \Tr(\alpha \cdot \beta)$ is a symmetric bilinear form over $K$.
\item For all $a \in K$ we have $\Tr_{\mc{A}}(a) = \dim(\mc{A}) \cdot a$.
\item\label{it:trace-zero-divisor} If $\alpha \in \mc{A}$ is such that $\alpha^D = 0$ for some $D \in \nat$, then $\Tr_{\mc{A}}(\alpha) = 0$.
\item\label{it:trace-isomorphism} If $\varphi : \mc{A} \to \mc{B}$ is a ring isomorphism, then $\Tr_{\mc{A}}(\alpha) = \Tr_{\mc{B}}(\varphi(\alpha))$ for $\alpha \in \mc{A}$.
\item\label{it:trace-product} If $\mc{C} = \mc{A} \times \mc{B}$ as a ring, then for $\alpha \in \mc{A}$ and $\beta \in \mc{B}$ we have $\Tr_{\mc{C}}((\alpha, \beta)) = \Tr_{\mc{A}}(\alpha) + \Tr_{\mc{B}}(\beta)$.
\end{enumerate}
\end{prop}
\begin{proof}
Exercise.
\end{proof}
%In view of part \eqref{it:trace-symmbil}, we can define the following
%\begin{defi}
%Let $\mc{A}$ be a finite-dimensional $K$-algebra.
%The \emph{trace form} is the symmetric bilinear form
%$$ \Tr : \mc{A} \times \mc{A} \to K : (\alpha, \beta) \mapsto \Tr(\alpha \cdot \beta).$$
%If $\charac(K) \neq 2$, we will also call the associated quadratic form $q_{\Tr}$ the trace form.
%\end{defi}
%Note that, if we want to compute the trace form of a given $K$-algebra $\mc{A}$, all we need to know is a basis $\alpha_1, \ldots, \alpha_n$ of $\mc{A}$, and for any $1 \leq i, j \leq n$, we need to know how to write $\alpha_i \alpha_j$ as a linear combination of the $\alpha_k$'s.
%
%\begin{eg}
%Consider the $\qq$-algebra $\mc{A} = \qq[i]$ with $i^2 = -1$.
%A basis for this algebra is given by $\lbrace 1,i \rbrace$.
%With respect to this basis, the matrices corresponding to $l_1$ and $l_i$ (as in \Cref{D:trace}) are
%$$ M_1 = \begin{bmatrix}
%1 & 0 \\ 0 & 1
%\end{bmatrix} \quad \text{and} \quad M_i = \begin{bmatrix}
%0 & -1 \\ 1 & 0
%\end{bmatrix}.$$
%For an arbitrary element $a + bi \in \mc{A}$ (with $a, b \in \qq$) we thus compute
%$$\Tr(a+bi) = a\Tr(1) + b\Tr(i) = a\Tr(M_1) + b\Tr(M_i) = 2a.$$
%It follows that the (quadratic) trace form is equal to
%$$ \Tr : \mc{A} \to \qq : a+bi \mapsto \Tr((a+bi)^2) = \Tr(a^2 - b^2 + 2abi) = 2a^2 - 2b^2, $$
%which is thus isometric to $\langle 2, -2 \rangle_{\qq} \cong \mbb{H}$.
%\end{eg}

\subsection{Hermite's method for counting real zeros}
We now explain how trace forms can be used to count the number of real zeros of a polynomial.

Assume throughout that $\charac(K) \neq 2$.
For a non-zero polynomial $f \in K[X]$ of degree $d$, denote by $\mc{A}_f$ the $d$-dimensional $K$-algebra $K[X]/(f(X))$.
Given another polynomial $g \in K[X]$, we denote its equivalence class in $\mc{A}_f$ by $\overline{g(X)}$ if it is clear from the context what $f$ is, or as $\overline{g(X)}^{[f]}$ if confusion is to be avoided.
\begin{defi}
Let $f, g \in K[X]$, $f \neq 0$.
The \emph{Hermite form}\index{Hermite form} $H(f, g)$ is the symmetric bilinear form
$$ H(f, g) : \mc{A}_f \times \mc{A}_f \to K : (\alpha, \beta) \mapsto \Tr_{\mc{A}_f}(\overline{g(X)} \cdot \alpha \cdot \beta).$$
We will also denote by $H(f, g)$ the associated quadratic form, i.e.~the form
$$ \mc{A}_f \to K : \alpha \mapsto \Tr_{\mc{A}_f}(\ovl{g(X)} \alpha^2).$$
\end{defi}

\begin{thm}\label{T:Hermite}
Let $f, g \in \rr[X]$ with $f \neq 0$.
Suppose that $m, n, p \in \nat$ such that
$$ H(f, g) \cong m \times \langle 1 \rangle_\rr \perp n \times \langle -1 \rangle_\rr \perp p \times \langle 0 \rangle_\rr.$$
Then
\begin{align*}
n + m &= \lvert \lbrace x \in \cc \mid f(x) = 0, g(x) \neq 0 \rbrace \rvert, \text{ and} \\
n - m &=  \lvert \lbrace x \in \rr \mid f(x) = 0, g(x) > 0 \rbrace \rvert - \lvert \lbrace x \in \rr \mid f(x) = 0, g(x) < 0 \rbrace \rvert.
\end{align*}
\end{thm}
Let us denote the quantity $n+m$ in \Cref{T:Hermite} by $C(f, g)$ and the quantity $n-m$ by $R(f, g)$, and note that we can compute these quantities just by knowing the coefficients of $f$ and $g$.
In particular, if $f$ and $g$ have coefficients in $\qq$ (or some other computable subfield of $\rr$), then one can algorithmically compute $R(f, g)$ and $C(f, g)$.

From \Cref{T:Hermite} we obtain in particular
\begin{displaymath}
C(f, 1) = \lvert \lbrace x \in \cc \mid f(x) = 0 \rbrace\rvert \text{ and } R(f, 1) =  \lvert \lbrace x \in \rr \mid f(x) = 0 \rbrace \rvert.
\end{displaymath}
In fact, one can count real roots of a polynomial $f$ with any finite number of side conditions, for example within a given interval.
\begin{cor}\label{C:Hermite}
Let $f \in \rr[X]$ non-zero, $m \in \nat^+$, $g_1, \ldots, g_m \in \rr[X]$.
Then
$$ \lvert \lbrace x \in \rr \mid f(x) = 0, g_1(x) > 0, \ldots, g_m(x) > 0 \rbrace \rvert = \frac{1}{2^m} \sum_{\alpha \in \lbrace 1, 2 \rbrace} R(f, g_1^{\alpha_1} \cdots g_m^{\alpha_m}). $$
\end{cor}
\begin{proof}
Exercise.
\end{proof}
\begin{eg}
Consider the polynomial $f(X) = X^3 - 3X + 1$. We use Hermite's method to determine how many real zeros this polynomial has, by computing the form $H(f, 1)$.

We consider the algebra $\mc{A}_f = \rr[X]/(X^3 - 3X + 1)$.
A basis for this algebra is given by $\lbrace 1, \ovl{X}, \ovl{X}^2 \rbrace$.
We have $\ovl{X}^3 = 3\ovl{X} - 1$ and $\ovl{X}^4 = 3\ovl{X}^2 - \ovl{X}$.
Thus, with respect to this basis, the matrices corresponding to $l_1$, $l_{\ovl{X}}$, and $l_{\ovl{X}^2}$ (as in \Cref{D:trace}) are
\begin{displaymath}
M_1 = \begin{bmatrix}
1 & 0 & 0 \\
0 & 1 & 0 \\
0 & 0 & 1
\end{bmatrix}, \quad M_{\ovl{X}} = \begin{bmatrix}
0 & 0 & -1 \\
1 & 0 & 3 \\
0 & 1 & 0
\end{bmatrix}, \quad \text{and} \quad M_{\ovl{X}^2} = \begin{bmatrix}
0 & -1 & 0 \\
0 & 3 & -1 \\
1 & 0 & 3
\end{bmatrix}.
\end{displaymath}
Denoting $\Tr_{\mc{A}_f}$ simply by $\Tr$, we conclude that $\Tr(1) = 3$, $\Tr(\ovl{X}) = 0$ and $\Tr(\ovl{X}^2) = 6$, and hence $\Tr(\ovl{X}^3) = \Tr(3\ovl{X} - 1) = -3$ and $\Tr(\ovl{X}^4) = \Tr(3\ovl{X}^2 - \ovl{X}) = 18$.
We thus compute that for arbitrary $a, b, c \in \rr$ one has
\begin{align*}
\Tr((a + b\ovl{X} + c\ovl{X}^2)^2) &= \Tr(a^2 + b^2\ovl{X}^2 + c^2\ovl{X}^4 + 2(ab\ovl{X} + ac\ovl{X}^2 + bc\ovl{X}^3)) \\
&= 3a^2 + 6b^2 + 18c^2 + 12ac - 6bc \\
&= (a + 2c)^2 + 2(c + \frac{1}{2}b)^2 + \frac{1}{2}b^2.
\end{align*}
From this, we see that $H(f, 1) \cong \langle 1, 2, \frac{1}{2} \rangle_\rr \cong 3 \times \langle 1 \rangle_\rr$, and by \Cref{T:Hermite} we infer that $f$ has $3 = R(f, 1)$ real zeros.
\end{eg}

\begin{lem}\label{L:Hermite-properties}
Let $f, f_1, f_2, g \in K[X]$ with $f, f_1, f_2 \neq 0$.
Let $d = \deg(f)$.
\begin{enumerate}
\item\label{it:coprime} If $f_1$ and $f_2$ are coprime, then
$$ H(f_1f_2, g) \cong H(f_1, g) \perp H(f_2, g).$$
\item\label{it:prime-power} For $k \in \nat^+$ we have
$$ H(f^k, g) \cong k \cdot H(f, g) \perp (d(k-1)) \times \langle 0 \rangle_K.$$
\item\label{it:linear} For $a \in K$ we have
$$ H(X - a, g) \cong \langle g(a) \rangle_K.$$
\item\label{it:quadratic} If $K = \rr$ and $a, b \in \rr$ with $b \neq 0$, then
$$ H((X-a)^2 + b^2, g) \cong \begin{cases}
\mbb{H}_\rr &\text{if } g(a+bi) \neq 0 \\
\langle 0, 0 \rangle_\rr &\text{if } g(a+bi) = 0
\end{cases}.$$
\end{enumerate}
\end{lem}
\begin{proof}
\eqref{it:coprime}: By the Chinese Remainder Theorem, we have that $$ \mc{A}_{f_1f_2} \to \mc{A}_{f_1} \times \mc{A}_{f_2} : \alpha \mapsto (\ovl{\alpha}^{[f_1]}, \ovl{\alpha}^{[f_2]}) $$
is a ring isomorphism, and in fact it is an isomorphism of $\rr$-algebras. By parts \eqref{it:trace-isomorphism} and \eqref{it:trace-product} of \Cref{P:trace-properties} we compute that for $\alpha \in \mc{A}_{f_1f_2}$ we have
$$ \Tr_{\mc{A}_{f_1f_2}}(\ovl{g(X)}\alpha^2) = \Tr_{\mc{A}_{f_1}}(\ovl{g(X)}^{[f_1]}(\ovl{\alpha}^{[f_1]})^2) + \Tr_{\mc{A}_{f_2}}(\ovl{g(X)}^{[f_2]}(\ovl{\alpha}^{[f_2]})^2). $$
From this the desired statement follows.

\eqref{it:prime-power}: We may assume $k > 1$, otherwise there is nothing to show. 
Since the polynomial $X^i f(X)^j$ for $0 \leq i < d$ and $0 \leq j < k$ has degree $i + dj < dk = \deg(f^k)$, we see that the set $\lbrace \ovl{X}^i \ovl{f(X)}^j \mid 0 \leq i < d, 0 \leq j < k \rbrace$ is linearly independent in $\mc{A}_{f^k}$, hence (since its cardinality $dk$ is the dimension of $\mc{A}_{f^k}$) a basis.
Let $W_1$ be the subspace of $\mc{A}_{f^k}$ spanned by $\ovl{X}^i$ for $0 \leq i < d$ and $W_2$ the subspace spanned by $\ovl{X}^i\ovl{f(X)}^j$ for $0 \leq i < d$ and $1 \leq j < k$.
We clearly have $\mc{A}_{f^k} = W_1 \oplus W_2$.
In view of \Cref{M-P:intrinsic-orth-sum} it remains to show that $W_1 \perp W_2$ with respect to $H(f^k, g)$, that $H(f^k, g)\vert_{W_1} \cong k \cdot H(f, g)$, and that $H(f^k, g)\vert_{W_2} \cong (d(k-1)) \times \langle 0 \rangle_K$.

To this end, note that, for any $\alpha \in \mc{A}_{f^k}$, we have $(\alpha \ovl{f(X)})^{k} = \alpha^k (\ovl{f(X)})^k = 0$, and thus by part \eqref{it:trace-zero-divisor} of \Cref{P:trace-properties} we have $\Tr_{\mc{A}_{f^k}}(\alpha\ovl{f(X)}) = 0$.
Since any element of $W_2$ is a multiple of $\ovl{f(X)}$, this shows both that $W_1 \perp W_2$ with respect to $H(f^k, g)$ and that $H(f^k, g)\vert_{W_2} \cong (d(k-1)) \times \langle 0 \rangle_K$.

Finally, to show that $H(f^k, g)\vert_{W_1} \cong k \cdot H(f, g)$, we need to show that, for $\alpha \in W_1$, $\Tr_{\mc{A}_{f^k}}(\alpha) = k\Tr_{\mc{A}_{f}}(\ovl{\alpha}^{[f]})$.
One verifies that, with respect to the (ordered) basis $1, \ovl{X}, \ldots, \ovl{X}^{d-1}, \ovl{f(X)}, \ovl{X}\ovl{f(X)}, \ldots, \ovl{X}^{d-1}\ovl{f(X)}^{k-1}$, the map $l_{\ovl{X}^i} : \mc{A}_{f^k} \to K$ for $0 \leq i < d$ (as in \Cref{D:trace}) has matrix of the block form
\begin{displaymath}
\begin{bmatrix}
M & 0 & 0 & \cdots & 0 \\
\ast & M & 0 & \cdots & 0 \\
\ast & \ast & M & \ddots & \vdots \\
\vdots & \vdots & \ddots & \ddots & 0 \\
\ast & \ast & \cdots & \ast & M
\end{bmatrix}
\end{displaymath}
where $M$ is the ($d \times d$) matrix of $l_{(\ovl{X}^i)^{[f]}} : \mc{A}_f \to K$ with respect to the basis $1, \ovl{X}, \ldots, \ovl{X}^{d-1}$, and $0$ denotes the $d \times d$ zero matrix.
Hence the desired statement follows from standard computation with block diagonal matrcies.

Parts \eqref{it:linear} and \eqref{it:quadratic} are left as exercise.
\end{proof}
\begin{proof}[Proof of \Cref{T:Hermite}]
We may assume without loss of generality that $f$ is monic.

Suppose first that $f = f_1 \cdot f_2$ for coprime monic $f_1, f_2 \in \rr[X]$ of degree at least $2$. By \eqref{it:coprime} of \Cref{L:Hermite-properties} we have that $H(f_1f_2, g) \cong H(f_1, g) \perp H(f_2, g)$.
Furthermore, an element $x \in \cc$ is a root of $f$ if and only if it is either a root of $f_1$ or a root of $f_2$.
We see that if the Theorem holds for $f_1$ and $f_2$, then it also holds for $f$.
Hence, we reduce to considering the case where $f = \tilde{f}^k$ for some irreducible polynomial $\tilde{f} \in \rr[X]$ and $k \in \nat$.

By \eqref{it:prime-power} we have $H(f, g) \cong kH(\tilde{f}, g) \perp (\deg(\tilde{f})(k-1)) \times \langle 0 \rangle_\rr$, and since $k \in \rr^{\times 2}$, this is further isometric to $H(\tilde{f}, g) \perp (\deg(\tilde{f})(k-1)) \times \langle 0 \rangle_\rr$.
Since an element $x \in \cc$ is a root of $f$ if and only if it is a root of $\tilde{f}$, we see that if the Theorem holds for $\tilde{f}$, then also for $f$.
We have thus reduced to considering the case where $f$ is an irreducible polynomial.

By the Fundamental Theorem of Algebra, the only irreducible polynomials in $\rr[X]$ are linear polynomials and irreducible quadratic polynomials.
For the linear polynomial $f(X) = X - a$ for $a \in \rr$, we have $H(f, g) = \langle g(a) \rangle_\rr$ by \Cref{L:Hermite-properties}\eqref{it:linear}, and thus
$$ H(f, g) \cong \begin{cases}
\langle 1 \rangle_{\rr} &\text{if } g(a) > 0, \\
\langle -1 \rangle_{\rr} &\text{if } g(a) < 0, \\
\langle 0 \rangle_{\rr} &\text{if } g(a) = 0.
\end{cases}$$
Since $a$ is the only root of $f$ in $\cc$, this establishes the Theorem in this case.

Finally, assume $f(X)$ is irreducible quadratic, i.e.~with negative discriminant.
After completing the square, we may write $f(X) = (X-a)^2 + b^2$ for $a, b \in \rr$, $b \neq 0$.
Note that the roots of $f(X)$ in $\cc$ are precisely $a+bi$ and $a-bi$; in particular, $f(X)$ has no real roots.
The Theorem now follows from \Cref{L:Hermite-properties}\eqref{it:quadratic}, using that $g(a + bi) = g(a-bi)$, since $g \in \rr[X]$.
\end{proof}

\subsection{Exercises}
\begin{enumerate}
\item Prove \Cref{P:trace-properties}, \Cref{C:Hermite}. Complete the proof of \Cref{L:Hermite-properties}.
\item Use Hermite's method to determine the number of real roots of the polynomial $f(X) = X^3 - 3X^2 - 2X + 5$ which are larger than $\sqrt[3]{2}$.
\end{enumerate}

\end{document}