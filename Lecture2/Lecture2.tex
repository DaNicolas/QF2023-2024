\documentclass[12pt, leqno, british]{amsart}
\usepackage[style=alphabetic, backend=biber]{biblatex}
\usepackage{a4, amsmath}
\usepackage{mathtools}
\usepackage{amssymb}
\usepackage{amsthm, amscd, mathdots}
\swapnumbers
\usepackage{enumerate}
\usepackage{hyperref}
\usepackage{cleveref}
\usepackage{csquotes}
\usepackage{color}
\usepackage{datetime}
\usepackage{xr, standalone, import}

\theoremstyle{definition}
\newtheorem{defi}{Definition}[subsection]
\theoremstyle{plain}
\newtheorem{prop}[defi]{Proposition}
\newtheorem{lem}[defi]{Lemma}
\newtheorem{thm}[defi]{Theorem}
\newtheorem{cor}[defi]{Corollary}
\newtheorem{ques}[defi]{Question}
\theoremstyle{remark}
\newtheorem{rem}[defi]{Remark}
\newtheorem{eg}[defi]{Example}
\newtheorem{egs}[defi]{Examples}

\newcommand{\mc}{\mathcal}
\newcommand{\mf}{\mathfrak}
\newcommand{\mbb}{\mathbb}
\newcommand{\nat}{\mbb N}
\newcommand{\cc}{\mathbb C}
\newcommand{\rr}{\mathbb R}
\newcommand{\qq}{\mbb Q}
\newcommand{\ovl}{\overline}
\newcommand{\ff}{\mbb F}
\newcommand{\zz}{\mbb Z}

\DeclareMathOperator{\charac}{char}
\DeclareMathOperator{\id}{id}
\DeclareMathOperator{\Frac}{Frac}
\DeclareMathOperator{\Ker}{Ker}
\DeclareMathOperator{\Img}{Im}
\DeclareMathOperator{\Trd}{Trd}
\DeclareMathOperator{\Tr}{Tr}
\DeclareMathOperator{\Nrd}{Nrd}
\DeclareMathOperator{\GL}{GL}
\DeclareMathOperator{\Gal}{Gal}
\DeclareMathOperator{\ord}{ord}
\DeclareMathOperator{\trdeg}{trdeg}
\DeclareMathOperator{\supp}{supp}
\DeclareMathOperator{\rad}{rad}
\DeclareMathOperator{\sign}{sign}
\newcommand{\disc}{\mathrm{d}}

\newcommand{\llangle}{\langle\!\langle}
\newcommand{\rrangle}{\rangle\!\rangle}
\addbibresource{../bibliography.bib}
\externaldocument[M-]{../Lecture-notes}

\author{Nicolas Daans}
\address{Charles University, Faculty of Mathematics and Physics, Department of Algebra, Sokolov\-sk\' a 83, 18600 Praha~8, Czech Republic.}
\email{nicolas.daans@matfyz.cuni.cz}

\begin{document}

\section{Lecture 2}
Let always $K$ be a field.
\begin{defi}
Let $(V, q)$ be a quadratic space.
If $W$ is a subspace of $V$, the quadratic space $(W, q\vert_W)$ is called a \emph{subform}\index{subform} of $(V, q)$.
By abuse of terminology, we will also call a quadratic space $(U, q')$ which is isometric to $(W, q\vert_W)$ for some subspace $W$ of $V$ a subform of $(V, q)$.
\end{defi}
In this lecture, we will get closer to a classification of quadratic spaces over a given field, by decomposing quadratic spaces as orthogonal sums of subforms with specific properties.

\subsection{Isotropic, totally isotropic, and hyperbolic forms}

Recall from \Cref{M-D:isotropic-represents-universal} the definition of an isotropic quadratic form.
\begin{defi}
Let $(V, q)$ be a quadratic space.
We call $(V, q)$ \emph{totally isotropic}\index{totally isotropic} if $q(v) = 0$ for all $v \in V$.
If $W$ is a subspace of $V$, we call $W$ totally isotropic if $(W, q\vert_W)$ is totally isotropic.
\end{defi}
Observe that a non-zero totally isotropic space is always singular.
\begin{prop}\label{P:radical-residue}
Assume $\charac(K) \neq 2$. Let $(V, q)$ be a quadratic space.
Then the map
$$ \ovl{q} : V/V^\perp \to K : \ovl{v} \mapsto q(v)$$
is a well-defined nonsingular quadratic form.
\end{prop}
\begin{proof}
The well-definedness follows from the fact that, for $v \in V$ and $w \in V^\perp$, one has $q(v + w) = q(v)$ by \Cref{M-P:nonsingular-polynomials}.
It is then easy to verify that the map is a quadratic form.

For the nonsingularity, consider $v \in V$ such that $\ovl{v} \neq 0$, i.e.~$v \not\in V^\perp$.
Then there exists $w \in V$ with $0 \neq \mf{b}_q(v, w) = \mf{b}_{\ovl{q}}(\ovl{v}, \ovl{w})$, whereby $\ovl{v} \not\in (V/V^\perp)^\perp$.
Hence $(V/V^\perp)^\perp = \lbrace 0 \rbrace$, and thus $(V/V^\perp, \ovl{q})$ is nonsingular.
\end{proof}

The following observation was already used implicitly in the proof of \Cref{M-P:diagonalisation}.
\begin{prop}\label{P:decomposition-totally-isotropic}
Assume $\charac(K) \neq 2$.
Let $(V, q)$ be a quadratic space.
Let $W$ be an orthogonal complement of $V^\perp$.
We have that $$(V, q) \cong (V^\perp, q\vert_{V^\perp}) \perp (W, q\vert_W),$$ that $(V^\perp, q\vert_{V^\perp})$ is totally isotropic, and that $(W, q\vert_W) \cong (V/V^\perp, \ovl{q})$.
\end{prop}
\begin{proof}
The first isometry is immediate form \Cref{M-P:intrinsic-orth-sum}.
The fact that $(V^\perp, q\vert_{V^\perp})$ is totally isotropic follows from \Cref{M-P:nonsingular-polynomials}.

Finally, consider the map
$$ \iota : W \to V/V^\perp : w \mapsto \ovl{w}.$$
Since $W \cap V^\perp = \lbrace 0 \rbrace$ we have that $\iota$ is injective, hence by comparing dimensions, $\iota$ is bijective.
Furthermore, by definition we have for $w \in W$ that $q(w) = \ovl{q}(\ovl{w}) = \ovl{q}(\iota(w))$.
Hence we have obtained the required isometry $(W, q\vert_W) \cong (V/V^\perp, \ovl{q})$.
\end{proof}
We can thus, in characteristic away from $2$, decompose any quadratic space into the orthogonal sum of a totally isotropic space and a nonsingular space, and this decomposition is unique up to isometry.

We now want to study nonsingular isotropic forms.
Nonsingular one-dimensional quadratic forms are always anisotropic.
\begin{defi}
We call the quadratic form $(K^2, q_f)$ with $f(X_1, X_2) = X_1 \cdot X_2$ the \emph{hyperbolic plane over $K$}\index{hyperbolic!plane} and denote it by $\mbb{H}_K$.
\end{defi}
\begin{prop}\label{P:hyperbolic-plane}
Let $(V, q)$ be a nonsingular quadratic space over $K$.
Let $v \in V \setminus \lbrace 0 \rbrace$ such that $q(v) = 0$.
Then there is a subspace $W \subseteq V$ with $v \in W$ such that $(W, q\vert_W)$ is isometric to $\mbb{H}_K$.
\end{prop}
\begin{proof}
Since $(V, q)$ is nonsingular, there exists $w \in V$ such that $a = \mf{b}_q(v, w) \neq 0$.
We may replace $w$ by $a^{-1}w$ and assume without loss of generality that $a = 1$.
Observe that $w \not\in Kv$, so that $W = Kv \oplus Kw$ is a $2$-dimensional subspace of $V$.
Consider the map
$$ \iota : K^2 \to W : (x, y) \mapsto xv + y(w - q(w)v).$$
Clearly this is a $K$-isomorphism of vector spaces.
We compute that, for $x, y \in K$, we have
\begin{align*}
q(\iota(x, y)) &= q(xv + y(w - q(w)v)) \\
&= (x-yq(w))^2 q(v) + y^2q(w) + \mf{b}_q((x - yq(w))v, yw) \\
&= 0 + y^2q(w) + (x - yq(w))y\mf{b}_q(v, w) = xy.
\end{align*}
Hence $(W, q\vert_W) \cong \mbb{H}_K$.
\end{proof}
In particular, it follows from \Cref{P:hyperbolic-plane} that the hyperbolic plane is, up to isometry, the only two-dimensional nonsingular isotropic quadratic form over $K$.
We also obtain the following
\begin{cor}\label{C:isotropic->universal}
Every nonsingular isotropic quadratic space is universal.
\end{cor}
\begin{proof}
We know from \Cref{M-E:hyp} that $\mbb{H}_K$ is universal.
But by \Cref{P:hyperbolic-plane} every nonsingular isotropic quadratic space contains $\mbb{H}_K$ as a subspace, hence is also universal.
\end{proof}
\begin{cor}\label{C:representation-theorem}
Let $(V, q)$ be a nonsingular quadratic space and $d \in K^\times$.
We have that $d \in D_K(q)$ if and only if $q \perp \langle -d \rangle_K$ is isotropic.
\end{cor}
\begin{proof}
Exercise.
\end{proof}

\begin{prop}\label{P:splitting-off}
Let $(V, q)$ be a nonsingular quadratic space, $W$ a nonsingular subspace of $V$.
Then $V = W \oplus W^\perp$,
$(V, q) \cong (W, q\vert_W) \perp (W^\perp, q\vert_{W^\perp}),$
and also $(W^\perp, q\vert_{W^\perp})$ is nonsingular.
\end{prop}
\begin{proof}
Since $(V, q)$ is nonsingular, we have $\dim W + \dim W^\perp = \dim V$ by \Cref{M-P:dim-duality}.
Since $(W, q\mid_W)$ is nonsingular, we further have $W \cap W^\perp = \lbrace 0 \rbrace$.
Hence, we obtain $V = W \oplus W^\perp$, and the natural induced $K$-isomorphism $V \to W \times W^\perp$ gives the required isometry $(V, q) \cong (W, q\vert_W) \perp (W^\perp, q\vert_{W^\perp})$; see \Cref{M-P:intrinsic-orth-sum}.

Finally, since $(W^\perp)^\perp = W$ by \Cref{M-P:dim-duality}, we obtain $(W^\perp)^\perp \cap W^\perp = W \cap W^\perp = \lbrace 0 \rbrace$, whereby $(W^\perp, q\vert_{W^\perp})$ is nonsingular.
\end{proof}
In the sequel, we will use the following notation: for a quadratic space $(V, q)$ and $n \in \nat$, we write
$$ n \times (V, q) = (V^n, \underbrace{q \perp \ldots \perp q}_{n \text{ times}}).$$
We will denote the described quadratic form on $V^n$ simply by $n \times q$.
By convention, $0 \times (V, q)$ denotes the unique zero-dimensional quadratic space over $K$.
\begin{prop}\label{P:hyperbolic-form}
Let $(V, q)$ be a nonsingular quadratic space, $n \in \nat$.
The following are equivalent.
\begin{enumerate}
\item\label{it:hyperbolic-form-1} $V$ contains a totally isotropic subspace of dimension $n$,
\item\label{it:hyperbolic-form-2} $V$ contains a subform isometric to $n \times \mbb{H}_K$.
\end{enumerate}
\end{prop}
\begin{proof}
For $n = 0$ there is nothing to show, assume from now on that $n \geq 1$.

Assume \eqref{it:hyperbolic-form-2}. 
Then $V$ has subspaces $W_1, \ldots, W_n$ such that $W_i \perp W_j$ and $W_i \cap W_j = \lbrace 0 \rbrace$ for any $i \neq j$ and such that $(W_i, q\vert_{W_i}) \cong \mbb{H}_K$.
Let $w_i \in W_i \setminus \lbrace 0 \rbrace$ be such that $q(w_i) = 0$.
Then $Kw_1 \oplus \ldots \oplus Kw_n$ is an $n$-dimensional totally isotropic subspace of $V$.

Conversely, assume \eqref{it:hyperbolic-form-1}.
We argue via induction on $n$ - recall that the case $n = 0$ is covered, so we assume $n \geq 1$.
Let $W$ be a totally isotropic subspace of $V$ of dimension $n$ and let $v \in W \setminus \lbrace 0 \rbrace$.
By \Cref{P:hyperbolic-plane} there exists $w \in V$ such that, for $W' = Kv \oplus Kw$, we have $(W', q\vert_{W'}) \cong \mbb{H}_K$.
By \Cref{P:splitting-off} this implies that $(V, q) \cong \mbb{H}_K \perp (U, q\vert_U)$ for $U=(W')^\perp$, and furthermore $(U, q\vert_U)$ is nonsingular.
Further, since $W \subseteq v^\perp$, we have
$$ U \cap W = (W')^\perp \cap W = v^\perp \cap w^\perp \cap W = w^\perp \cap W $$
whereby $\dim(U \cap W) \geq n-1$.
Hence $(U, q\vert_U)$ contains a totally isotropic subspace $U \cap W$ of dimension $n-1$.
The statement now follows by the induction hypothesis.
\end{proof}
\begin{cor}\label{C:hyperbolic-form}
Let $(V, q)$ be a nonsingular quadratic space of dimension $2n$, where $n \in \nat$.
The following are equivalent.
\begin{enumerate}
\item $V$ contains a totally isotropic subspace of dimension $n$,
\item $(V, q) \cong n \times \mbb{H}_K$.
\end{enumerate}
\end{cor}
\begin{defi}
We say that a nonsingular quadratic space of dimension $2n$ (for some $n \in \nat$) is \emph{hyperbolic}\index{hyperbolic!space} if it contains a totally isotropic subspace of dimension $n$.

Given a quadratic space $(V, q)$, we define the \emph{Witt index}\index{Witt index} of $(V, q)$ to be the maximal possible dimension of a totally isotropic subspace $W$ of $(V, q)$ with $V^\perp \cap W = \lbrace 0 \rbrace$.
We denote it by $i_W(V, q)$, or simply $i_W(q)$.
If $\charac(K) \neq 2$, $i_W(V, q)$ is also the maximal possible dimension of a totally isotropic subspace of $V/V^\perp$.
\end{defi}

\begin{prop}\label{P:q-q-hyperbolic}
Let $(V, q)$ be a nonsingular quadratic space. Then $(V, q) \perp (V, -q)$ is hyperbolic.
\end{prop}
\begin{proof}
Let $n = \dim V$. Then $\dim (V \times V) = 2n$.
Let $W = \lbrace (v, v) \in V \times V \mid v \in V \rbrace$. Then $W$ is a subspace of $V \times V$ of dimension $n$, and it is a totally isotropic subspace of $(V, q) \perp (V, -q)$, since for any $v \in V$ we have $(q \perp -q)(v, v) = q(v) - q(v) = 0$.

Since $(V, q) \perp (V, -q)$ is nonsingular (by \Cref{M-P:orth-sum}) and has a totally isotropic subspace of dimension $n$, it is hyperbolic.
\end{proof}

\subsection{Witt's Theorems}
We are now in a position to prove the two most important structure theorems on quadratic forms, named after Ernst Witt.
We will prove them, as Witt did in the 1930'ies, under the assumption that $\charac(K) \neq 2$.
Versions in characteristic 2 exist and can be proven with extra assumptions and a lot more work, see \autocite[Section 8]{ElmanKarpenkoMerkurjev}.
\begin{lem}\label{L:O(q)-transitive}
Assume that $\charac(K) \neq 2$.
Let $(V, q)$ be a quadratic space, and let $v, w \in V$ be such that $q(v) = q(w) \neq 0$.
There exists an isometry $\tau: (V, q) \to (V, q)$ such that $\tau(x) = y$.
\end{lem}
\begin{proof}
One computes that $q(v + w) + q(v - w) = 4q(v) \neq 0$, so at least one of $q(v+w)$ and $q(v-w)$ is non-zero.
Replacing $w$ by $-w$ if necessary, we may assume that $q(v-w) \neq 0$.
Now consider the map
$$\tau : V \to V : u \mapsto u - \frac{\mf{b}_q(u, v-w)}{q(v-w)}(v-w).$$
One verifies that $\tau$ gives an isometry $(V, q) \to (V, q)$, and that $\tau(v) = w$, as desired; see Exercise \eqref{ex-reflections}.
\end{proof}

\begin{thm}[Witt Cancellation Theorem]\label{T:Witt-Cancellation}
Assume $\charac(K) \neq 2$.
Let $(V, q)$, $(V_1, q_1)$ and $(V_2, q_2)$ be quadratic spaces.
If $(V, q) \perp (V_1, q_1) \cong (V, q) \perp (V_2, q_2)$, then $(V_1, q_1) \cong (V_2, q_2)$.
\end{thm}
\begin{proof}
We first reduce to the case where all involved quadratic spaces are nonsingular.
To this end, use \Cref{P:decomposition-totally-isotropic} to write $(V, q) \cong (V^\perp, q\vert_{V^\perp}) \perp (W, q\vert_W)$, $(V_1, q_1) \cong (V_1^\perp, q_1\vert_{V_1^\perp}) \perp (W_1, q\vert_{W_1})$ and $(V_2, q_2) \cong (V_2^\perp, q_2\vert_{V_2^\perp}) \perp (W_2, q\vert_{W_2})$ where $q\vert_W$, $q_1\vert_{W_1}$ and $q_2\vert_{W_2}$ are nonsingular.
The hypothesis can be rewritten as
\begin{align*}
&((V \perp V_1)^\perp, (q \perp q_1)\vert_{(V \perp V_1)^\perp}) \perp (W \perp W_1, (q \perp q_1)\vert_{W \perp W_1}) \\
\cong\enspace &((V \perp V_2)^\perp, (q \perp q_2)\vert_{(V \perp V_2)^\perp}) \perp (W \perp W_2, (q \perp q_2)\vert_{W \perp W_2}),
\end{align*}
using that $V^\perp \perp V_1^\perp = (V \perp V_1)^\perp$ and similarly $V^\perp \perp V_2^\perp = (V \perp V_2)^\perp$.
We further have by \Cref{M-P:orth-sum} that $(W \perp W_1, (q \perp q_1)\vert_{W \perp W_1})$ and $(W \perp W_2, (q \perp q_2)\vert_{W \perp W_2})$ are nonsingular.
In view of \Cref{P:decomposition-totally-isotropic} we have
\begin{align*}
&(W \perp W_1, (q \perp q_1)\vert_{W \perp W_1}) \cong ((V \perp V_1)/(V \perp V_1)^\perp, \overline{q \perp q_1}) \\
\cong\enspace &((V \perp V_2)/(V \perp V_2)^\perp, \overline{q \perp q_2}) \cong (W \perp W_2, (q \perp q_2)\vert_{W \perp W_2}),
\end{align*}
We conclude that we may assume for the remainder of the proof that $(V, q)$, $(V_1, q_1)$ and $(V_2, q_2)$ are nonsingular.

By \Cref{M-C:diagonalisation} we may assume that $(V, q) \cong \langle a_1, \ldots, a_n \rangle$ for some $a_1, \ldots, a_n \in K^\times$.
By inducting on $n$, we reduce to the situation $n = 1$.
Let $\iota : \langle a \rangle_K \perp (V_1, q_1) \to \langle a \rangle_K \perp (V_2, q_2)$ be an isometry.
Let $v = \iota(1, 0)$.
We have $(\langle a \rangle_K \perp q_2)(v) = (\langle a \rangle_K \perp q_1)(1 , 0) = a\cdot 1^2 = a = (\langle a \rangle_K \perp q_2)(1, 0)$.

By \Cref{L:O(q)-transitive} there exists an isometry $\tau : \langle a \rangle_K \perp (V_2, q_2) \to \langle a \rangle_K \perp (V_2, q_2)$ with $\tau(v) = (1, 0)$.
Thus, $\tau \circ \iota$ gives an isometry $\langle a \rangle_K \perp (V_1, q_1) \to \langle a \rangle_K \perp (V_2, q_2)$ mapping $(1, 0)$ to $(1, 0)$.
Furthermore, since $(K \times \lbrace 0 \rbrace) \perp (\lbrace 0 \rbrace \times V_1)$ (in $(K \times V_1, \langle a \rangle_K \perp q_1)$) and isometries preserve orthogonality, we obtain $(K \times \lbrace 0 \rbrace) \perp (\tau \circ \iota)(\lbrace 0 \rbrace \times V_1)$ (in $(K \times V_2, \langle a \rangle_K \perp q_2)$).
So, we must have $(\tau \circ \iota)(\lbrace 0 \rbrace \times V_1) = \lbrace 0 \rbrace \times V_2$, whereby $\tau \circ \iota$ induces an isometry $(V_1, q_1) \to (V_2, q_2)$, as desired.
\end{proof}

\begin{thm}[Witt Decomposition Theorem]\label{T:Witt-Decomposition}
Assume $\charac(K) \neq 2$.
Let $(V, q)$ be a quadratic space.
There exist quadratic spaces $(V_t, q_t)$, $(V_h, q_h)$ and $(V_a, q_a)$ such that
$$ (V, q) \cong (V_t, q_t) \perp (V_h, q_h) \perp (V_a, q_a)$$ where
\begin{itemize}
\item $(V_t, q_t)$ is totally isotropic,
\item $(V_h, q_h)$ is hyperbolic (or zero),
\item $(V_a, q_a)$ is anisotropic.
\end{itemize}
Furthermore, each of these spaces is determined up to isometry by $(V, q)$.
In fact, $(V_t, q_t)$ is the unique totally isotropic space of dimension $\dim V^\perp$, and $(V_h, q_h)$ is the unique hyperbolic space of dimension $2i_W(q)$.
\end{thm}
\begin{proof}
We first prove the existence of the required spaces.
By \Cref{P:decomposition-totally-isotropic} we can write $(V, q) \cong (V_t, q_t) \perp (V', q')$ where $(V_t, q_t)$ is totally isotropic of dimension $\dim V^\perp$ and $(V', q')$ is nonsingular.
Let $m = i_W(V, q)$.
By \Cref{P:hyperbolic-form} and \Cref{P:splitting-off} we can write $(V', q') \cong (V_h, q_h) \perp (V_a, q_a)$ where $(V_h, q_h)$ is hyperbolic of dimension $2m$.
$(V_a, q_a)$ must be nonsingular, and in fact it is anisotropic, since otherwise one could find a totally isotropic subspace of $(V', q')$ of dimension $m+1$, contradicting the choice of $m$.
This concludes the existence part of the proof.

For the uniqueness, assume that
$$ (V, q) \cong (V_t, q_t) \perp (V_h, q_h) \perp (V_a, q_a) \cong (V_t', q_t') \perp (V_h', q_h') \perp (V_a', q_a')$$
where $(V_t', q_t')$ is totally singular, $(V_h', q_h')$ is hyperbolic, and $(V_a', q_a')$ is anisotropic.
Since $(V_t', q_t')$ is totally isotropic and $(V_h', q_h') \perp (V_a', q_a')$ is nonsingular, we must have
$$ \dim V_t' = \dim V^\perp = \dim V_t.$$
Since $(V_t, q_t)$ and $(V_t', q_t')$ are totally isotropic of the same dimension, they must be isometric.
By \Cref{T:Witt-Cancellation} we obtain that $(V_h, q_a) \perp (V_a, q_a) \cong (V_h', q_h') \perp (V_a', q_a')$.
Similarly, since $(V_h', q_h')$ is hyperbolic and $(V_a', q_a')$ is anisotropic, we must have $\dim V_h' = 2m = \dim V_h$, whereby $(V_h, q_h)$ and $(V_h', q_h')$ are hyperbolic forms of the same dimension and hence isometric.
Finally, applying \Cref{T:Witt-Cancellation} again, we obtain $(V_a, q_a) \cong (V_a', q_a')$.
\end{proof}

\subsection{Exercises}
\begin{enumerate}
%\item For a quadratic space $(V, q)$, define the \emph{quadratic radical of $q$} as the set
%$$ \rad(q) = \lbrace v \in V^\perp \mid q(v) = 0 \rbrace.$$
%Note that, if $\charac(K) \neq 2$, then $\rad(q) = V^\perp$.
%Show that \Cref{P:radical-residue} does not hold as stated when $\charac(K) = 2$, but still holds with $\charac(K) = 2$ if one replaces $V^\perp$ by $\rad(q)$.
\item Complete the proof of \Cref{C:representation-theorem}.
\item\label{ex-reflections} Let $(V, q)$ be a quadratic space, and consider for $v \in V$ with $q(v) \neq 0$ the map
$$ \tau_v : V \to V : w \mapsto w - \frac{\mf{b}_q(w, v)}{q(v)}v.$$
Show the following for any $v \in V$ with $q(v) \neq 0$:
\begin{enumerate}
\item $\tau_v$ is an isometry $(V, q) \to (V, q)$,
\item $\tau_v(v) = -v$, and for $w \in v^\perp$ we have $\tau_v(w) = w$,
\item If $w \in V$ is such that $q(v) = q(w)$ and $q(v-w) \neq 0$, then $\tau_{v-w}(v) = w$.
\end{enumerate}
\item Show that the following are equivalent for a field $K$ with $\charac(K) \neq 2$:
\begin{enumerate}
\item Any two nonsingular quadratic spaces over $K$ of the same dimension are isometric.
\item Every element of $K$ is a square.
\end{enumerate}
\item\label{ex:basis-isotropic-vectors} Let $(V, q)$ be a nonsingular isotropic space.
Show that $V$ has a basis consisting of isotropic vectors.
\item Let $(V, q)$ be a nonsingular quadratic space, set $n = \dim V$ and $m = i_W(q)$.
Show that every subform of $(V, q)$ of dimension greater than $n - m$ is isotropic.
\item Assume $\charac(K) \neq 2$ and let $(V_1, q_1)$ and $(V_2, q_2)$ be nonsingular quadratic spaces over $K$.
Show that $(V_2, q_2)$ is a subform of $(V_1, q_1)$ if and only if $i_W((V_1, q_1) \perp (V_2, -q_2)) \geq \dim V_2$.
\item Let $K = \ff_2$, the field with two elements. Consider the quadratic form
\begin{displaymath}
q : K^2 \to K : (x, y) \mapsto x^2 + xy + y^2.
\end{displaymath}
Show that $q \perp \langle 1 \rangle_K \cong \mbb{H}_K \perp \langle 1 \rangle_K$, but $q \not\cong \mbb{H}_K$.
Conclude that \Cref{T:Witt-Cancellation} does not hold as stated without the assumption $\charac(K) \neq 2$.
\end{enumerate}
\end{document}