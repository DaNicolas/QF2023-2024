\documentclass[12pt, leqno, british]{amsart}
\usepackage[style=alphabetic, backend=biber]{biblatex}
\usepackage{a4, amsmath}
\usepackage{mathtools}
\usepackage{amssymb}
\usepackage{amsthm, amscd, mathdots}
\swapnumbers
\usepackage{enumerate}
\usepackage{hyperref}
\usepackage{cleveref}
\usepackage{csquotes}
\usepackage{color}
\usepackage{datetime}
\usepackage{xr, standalone, import}

\theoremstyle{definition}
\newtheorem{defi}{Definition}[subsection]
\theoremstyle{plain}
\newtheorem{prop}[defi]{Proposition}
\newtheorem{lem}[defi]{Lemma}
\newtheorem{thm}[defi]{Theorem}
\newtheorem{cor}[defi]{Corollary}
\newtheorem{ques}[defi]{Question}
\theoremstyle{remark}
\newtheorem{rem}[defi]{Remark}
\newtheorem{eg}[defi]{Example}
\newtheorem{egs}[defi]{Examples}

\newcommand{\mc}{\mathcal}
\newcommand{\mf}{\mathfrak}
\newcommand{\mbb}{\mathbb}
\newcommand{\nat}{\mbb N}
\newcommand{\cc}{\mathbb C}
\newcommand{\rr}{\mathbb R}
\newcommand{\qq}{\mbb Q}
\newcommand{\ovl}{\overline}
\newcommand{\ff}{\mbb F}
\newcommand{\zz}{\mbb Z}

\DeclareMathOperator{\charac}{char}
\DeclareMathOperator{\id}{id}
\DeclareMathOperator{\Frac}{Frac}
\DeclareMathOperator{\Ker}{Ker}
\DeclareMathOperator{\Img}{Im}
\DeclareMathOperator{\Trd}{Trd}
\DeclareMathOperator{\Tr}{Tr}
\DeclareMathOperator{\Nrd}{Nrd}
\DeclareMathOperator{\GL}{GL}
\DeclareMathOperator{\Gal}{Gal}
\DeclareMathOperator{\ord}{ord}
\DeclareMathOperator{\trdeg}{trdeg}
\DeclareMathOperator{\supp}{supp}
\DeclareMathOperator{\rad}{rad}
\DeclareMathOperator{\sign}{sign}
\newcommand{\disc}{\mathrm{d}}

\newcommand{\llangle}{\langle\!\langle}
\newcommand{\rrangle}{\rangle\!\rangle}
\addbibresource{../bibliography.bib}
\externaldocument[M-]{../Lecture-notes}

\author{Nicolas Daans}
\address{Charles University, Faculty of Mathematics and Physics, Department of Algebra, Sokolov\-sk\' a 83, 18600 Praha~8, Czech Republic.}
\email{nicolas.daans@matfyz.cuni.cz}

\begin{document}

\section{Lecture 3}
\subsection{Tensor products of symmetric bilinear spaces}
In this section, we will define the tensor product (sometimes called Kronecker product) of two symmetric bilinear spaces.
First, we define the tensor product of two $K$-vector spaces.

Let $V$ and $W$ be $K$-vector spaces.
Denote by $K^{(V \times W)}$ the free $K$-vector space over the set $V \times W$.
That is, for each $(v, w) \in V \times W$ we fix an element $e_{(v, w)} \in K^{(V \times W)}$, and then $\lbrace e_{(v, w)} \mid (v, w) \in V \times W \rbrace$ is a basis of $K^{(V \times W)}$.
Let $A$ be the subspace of $K^{(V \times W)}$ generated by elements of the form
\begin{displaymath}
e_{(v+av', w)} - e_{(v, w)} - ae_{(v', w)} \quad\text{or}\quad e_{(v,w+aw')} - a_{(v, w)} - ae_{(v, w')}
\end{displaymath}
for $v, v' \in V$, $w, w' \in W$ and $a \in K$.
\begin{defi}
With the notations from above, we call the quotient space $K^{(V \times W)}/A$ the \emph{tensor product of V and W}\index{tensor product!of vector spaces}, which we denote by $V \otimes W$ - or $V \otimes_K W$ if we want to stress the underlying field.
For $v \in V$ and $w \in W$ we denote by $v \otimes w$ the class of $e_{(v, w)}$ in this quotient space.
We call an element of $V \otimes_K W$ of the form $v \otimes w$ for $v \in V$ and $w \in W$ an \emph{elementary tensor}\index{elementary tensor}.
\end{defi}
\begin{rem}
Be careful! Not every element of $V \otimes W$ is of the form $v \otimes w$ for $v \in V$ and $w \in W$, i.e.~not every element of $V \otimes W$ is an elementary tensor.
However, every element of $V \otimes W$ is a sum of elementary tensors - although this decomposition is not unique.
\end{rem}
The tensor product $V \otimes W$ is best understood through the following fundamental property.
\begin{prop}[Universal property of tensor products]\label{P:tensor-product-universal-property}
Let $V$ and $W$ be $K$-vector spaces.
The map $V \times W \to V \otimes W : (v, w) \mapsto v \otimes w$ is a bilinear map, and its image generates $V \otimes W$.

For any $K$-vector space $U$ and any bilinear map $B : V \times W \to U$,
there exists a unique linear map $\ovl{B} : V \otimes W \to U$ such that $B(v, w) = \ovl{B}(v \otimes w)$ for all $v \in V$, $w \in W$.
\end{prop}
\begin{proof}
The bilinearity of the map $V \times W \to V \otimes W : (v, w) \mapsto v \otimes w$ follows from the construction of $V \otimes W$: we have for any $v_1, v_2 \in V$, $w_1, w_2 \in W$ and $a, b \in K$ that
$$ (v_1 + av_2) \otimes (w_1 + bw_2) = (v_1 \otimes w_1) + a(v_2 \otimes w_1) + b(v_1 \otimes w_2) + ab(v_2 \otimes w_2).$$
The image of the map consists of elementary tensors, which by construction generate $V \otimes W$.

Now consider any bilinear map $B : V \times W \to U$.
Since $\lbrace e_{(v, w)} \mid (v, w) \in V \times W \rbrace$ form a basis of $K^{(V \times W)}$, there is a unique $K$-linear map $\hat{B} : K^{(V \times W)} \to U$ mapping $e_{(v, w)}$ to $B(v, w)$ for $(v, w) \in V \times W$.
By the bilinearity of $B$, we compute that for $v_1, v_2 \in V$ and $w_1, w_2 \in W$ we have
\begin{align*}
\hat{B}(e_{(v_1 + av_2, w_1 + bw_2)}) &= B(v_1 + av_2, w_1 + bw_2) \\
&= B(v_1, w_1) + aB(v_2, w_1) + bB(v_1, w_2) + abB(v_2, w_2) \\
&= \hat{B}(e_{(v_1, w_1)} + ae_{(v_2, w_1)} + be_{(v_1, w_2)} + abe_{(v_2, w_2)}).
\end{align*}
As such, $\Ker(\hat{B})$ contains all elements given as generators for the subspace $A$ of $K^{(V \times W)}$, whereby $A \subseteq \Ker(\hat{B})$.
Recalling that $V \otimes W = K^{(V \times W)}/A$, we conclude that there exists a unique linear map $\overline{B} : V \otimes W \to U$ such that $\overline{B}(v \otimes w) = \hat{B}(e_{(v, w)}) = B(v, w)$ for all $(v, w) \in V \times W$.
\end{proof}
\begin{prop}\label{P:tensor-product-properties}
Let $U$, $V$ and $W$ be $K$-vector spaces.
The tensor product satisfies the following properties.
\begin{enumerate}
\item There is a unique $K$-isomorphism $V \otimes W \to W \otimes V$ such that $v \otimes w \mapsto w \otimes v$ for $v \in V$ and $w \in W$.
\item There is a unique $K$-isomorphism $(U \otimes V) \otimes W \to U \otimes (V \otimes W)$ such that $(u \otimes v) \otimes w \mapsto u \otimes (v \otimes w)$ for $u \in U$, $v \in V$ and $w \in W$.
\item There is a unique $K$-isomorphism $(U \times V) \otimes W \to (U \otimes W) \times (V \otimes W)$ such that $((u , v), w) \mapsto ((u \otimes w), (v \otimes w))$ for $u \in U$, $v \in V$ and $w \in W$.
\item Let $\mf{B}_V$ and $\mf{B}_W$ be bases for $V$ and $W$ respectively.
Then
$$ \lbrace v \otimes w \mid v \in \mf{B}_B, w \in \mf{B}_W \rbrace $$
is a basis for $V \otimes W$.
In particular, $\dim(V \otimes W) = \dim(V)\dim(W)$.
\end{enumerate}
\end{prop}
\begin{proof}
Each of these can be proven by making use of \Cref{P:tensor-product-universal-property}.
\end{proof}
We can now define the tensor product of symmetric bilinear spaces.
\begin{prop}\label{P:tensor-product-SBS}
Let $(V_1, B_1)$ and $(V_2, B_2)$ be symmetric bilinear spaces.
There exists a unique $K$-bilinear form $B$ on $V_1 \otimes V_2$ such that
$$ B(v_1 \otimes v_2, w_1 \otimes w_2) = B_1(v_1,w_1) \cdot B_2(v_2, w_2) $$
for all $v_1, w_1 \in V_1$ and $v_2, w_2 \in V_2$.
%If $B_1$ and $B_2$ are non-degenerate, then so is $B$.
\end{prop}
\begin{proof}
The uniqueness is clear, since $V \otimes W$ is generated by elementary tensors; furthermore, since such a bilinear map would by definition be symmetric on elementary tensors, it is automatically symmetric.
It thus suffices to show the existence of such a bilinear map $B$.

Consider first for $(v_1, v_2) \in V_1 \times V_2$ the map
$$ V_1 \times V_2 \to K : (w_1, w_2) \mapsto B_1(v_1, w_1) \cdot B_2(v_2, w_2). $$
This map is bilinear, hence by \Cref{P:tensor-product-universal-property} induces a linear map $B_{(v_1, v_2)} : V_1 \otimes V_2 \to K$ such that $B_{(v_1, v_2)}(w_1 \otimes w_2) = B_1(v_1, w_1) \cdot B_2(v_2, w_2)$ for $w_1 \in V_1$ and $w_2 \in V_2$.
The map
$$ B^\ast : V_1 \times V_2 \to (V_1 \otimes V_2)^\ast : (v_1, v_2) \mapsto B_{(v_1, v_2)}$$
is also bilinear, hence, again by \Cref{P:tensor-product-universal-property}, it induces a linear map $\ovl{B^\ast} : V_1 \otimes V_2 \to (V_1 \otimes V_2)^\ast$ such that $\ovl{B^\ast}(v_1 \otimes v_2) = B_{(v_1, v_2)}$ for $(v_1, v_2) \in V_1 \times V_2$.

Finally, consider the bilinear map
$$ B : (V_1 \otimes V_2) \times (V_1 \otimes V_2) : (\alpha, \beta) \mapsto \ovl{B^\ast}(\alpha)(\beta).$$
We compute that, for $v_1, w_1 \in V_1$ and $v_2, w_2 \in V_2$, we have
\begin{align*}
B(v_1 \otimes v_2, w_1 \otimes w_2) &= \ovl{B^\ast}(v_1 \otimes v_2)(w_1 \otimes w_2) \\
&= B_{(v_1, v_2)}(w_1 \otimes w_2) = B_1(v_1, w_1) \cdot B_2(v_2, w_2).
\end{align*}
Hence, $B$ is as desired.
\end{proof}
\begin{defi}
Given symmetric bilinear spaces $(V_1, B_1)$ and $(V_2, B_2)$, we call the symmetric bilinear space constructed in \Cref{P:tensor-product-SBS} the \emph{tensor product}\index{tensor product!of symmetric bilinear spaces} of $(V_1, B_1)$ and $(V_2, B_2)$.
We denote it by $(V_1 \otimes V_2, B_1 \otimes B_2)$.

Over fields of characteristic different from $2$, we will also consider the tensor product of quadratic spaces; this is by definition the quadratic space corresponding to the tensor product of the underlying symmetric bilinear spaces, see \Cref{P:quadratic-vs-SBS}.
That is, for quadratic spaces $(V_1, q_1)$ and $(V_2, q_2)$, we define
$$ q_1 \otimes q_2 : V_1 \otimes V_2 \to K : \alpha \mapsto \frac{(B_{q_1} \otimes B_{q_2})(\alpha)}{4}. $$
\end{defi}
In the following proposition stating some computation rules, in the interest of brevity, we represent a quadratic space just by its quadratic form.
\begin{prop}\label{P:tensor-product-properties-QF}
Assume $\charac(K) \neq 2$.
For quadratic forms $q_1, q_2, q_3$ over $K$ we have
\begin{align*}
q_1 \otimes q_2 &\cong q_2 \otimes q_1 \\
(q_1 \otimes q_2) \otimes q_3 &\cong q_1 \otimes (q_2 \otimes q_3) \\
(q_1 \perp q_2) \otimes q_3 &\cong (q_1 \otimes q_3) \perp (q_2 \otimes q_3)
\end{align*}
\end{prop}
\begin{proof}
Each of these follows by checking that the isomorphism of vector spaces established in \Cref{P:tensor-product-properties} induces isometries of quadratic (/symmetric bilinear) spaces.
\end{proof}
\begin{cor}\label{C:tensor-product-diagonal}
Assum $\charac(K) \neq 2$.
Let $m, n \in \nat$ and let $a_1, \ldots, a_m, b_1, \ldots, b_n \in K$.
We have
$$ \langle a_1, \ldots, a_m \rangle_K \otimes \langle b_1, \ldots b_n \rangle_K \cong \langle a_1b_1, \ldots, a_ib_j, \ldots, a_mb_n \rangle_K $$
\end{cor}
\begin{proof}
This follows by \Cref{P:tensor-product-properties-QF} and the easy observation that $\langle a \rangle_K \otimes \langle b \rangle_K \cong \langle ab \rangle_K$ for $a, b \in K$.
\end{proof}
\begin{cor}\label{C:tensor-product-nonsingular}
Assume $\charac(K) \neq 2$.
Let $(V_1, q_1), (V_2, q_2)$ be nonsingular quadratic spaces.
Then $(V_1 \otimes V_2, q_1 \otimes q_2)$ is nonsingular.
\end{cor}
\begin{proof}
By \Cref{C:diagonalisation} and \Cref{P:diagforms-singular} both $(V_1, q_1)$ and $(V_2, q_2)$ are isometric to diagonal forms where all entries are non-zero.
By \Cref{C:tensor-product-diagonal} the same holds for $(V_1 \otimes V_2, q_1 \otimes q_2)$, whence this form is also nonsingular.
\end{proof}
\begin{cor}\label{C:tensor-product-hyperbolic}
Assume $\charac(K) \neq 2$.
Let $(V, q)$ be a nonsingular quadratic space.
Then $(V, q) \otimes \mbb{H}_K$ is hyperbolic.
\end{cor}
\begin{proof}
We have $\mbb{H}_K \cong \langle 1, -1 \rangle_K$ (see \Cref{E:hyperbolic-plane-isometry}).
Hence, by \Cref{P:tensor-product-properties-QF},
$$ (V, q) \otimes \mbb{H}_K \cong (V, q) \otimes \langle 1, -1 \rangle_K \cong (V, q) \perp (V, -q) $$
which is hyperbolic by \Cref{P:q-q-hyperbolic}.
\end{proof}

\subsection{Exercises}
\begin{enumerate}
\item Prove \Cref{P:tensor-product-properties} and \Cref{P:tensor-product-properties-QF}.
\end{enumerate}
\end{document}