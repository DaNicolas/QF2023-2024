\documentclass[12pt, leqno, british]{amsart}
\usepackage[style=alphabetic, backend=biber]{biblatex}
\usepackage{a4, amsmath}
\usepackage{mathtools}
\usepackage{amssymb}
\usepackage{amsthm, amscd, mathdots}
\swapnumbers
\usepackage{enumerate}
\usepackage{hyperref}
\usepackage{cleveref}
\usepackage{csquotes}
\usepackage{color}
\usepackage{datetime}
\usepackage{xr, standalone, import}

\theoremstyle{definition}
\newtheorem{defi}{Definition}[subsection]
\theoremstyle{plain}
\newtheorem{prop}[defi]{Proposition}
\newtheorem{lem}[defi]{Lemma}
\newtheorem{thm}[defi]{Theorem}
\newtheorem{cor}[defi]{Corollary}
\newtheorem{ques}[defi]{Question}
\theoremstyle{remark}
\newtheorem{rem}[defi]{Remark}
\newtheorem{eg}[defi]{Example}
\newtheorem{egs}[defi]{Examples}

\newcommand{\mc}{\mathcal}
\newcommand{\mf}{\mathfrak}
\newcommand{\mbb}{\mathbb}
\newcommand{\nat}{\mbb N}
\newcommand{\cc}{\mathbb C}
\newcommand{\rr}{\mathbb R}
\newcommand{\qq}{\mbb Q}
\newcommand{\ovl}{\overline}
\newcommand{\ff}{\mbb F}
\newcommand{\zz}{\mbb Z}

\DeclareMathOperator{\charac}{char}
\DeclareMathOperator{\id}{id}
\DeclareMathOperator{\Frac}{Frac}
\DeclareMathOperator{\Ker}{Ker}
\DeclareMathOperator{\Img}{Im}
\DeclareMathOperator{\Trd}{Trd}
\DeclareMathOperator{\Tr}{Tr}
\DeclareMathOperator{\Nrd}{Nrd}
\DeclareMathOperator{\GL}{GL}
\DeclareMathOperator{\Gal}{Gal}
\DeclareMathOperator{\ord}{ord}
\DeclareMathOperator{\trdeg}{trdeg}
\DeclareMathOperator{\supp}{supp}
\DeclareMathOperator{\rad}{rad}
\DeclareMathOperator{\sign}{sign}
\newcommand{\disc}{\mathrm{d}}

\newcommand{\llangle}{\langle\!\langle}
\newcommand{\rrangle}{\rangle\!\rangle}
\addbibresource{../bibliography.bib}
\externaldocument[M-]{../Lecture-notes}

\author{Nicolas Daans}
\address{Charles University, Faculty of Mathematics and Physics, Department of Algebra, Sokolov\-sk\' a 83, 18600 Praha~8, Czech Republic.}
\email{nicolas.daans@matfyz.cuni.cz}

\begin{document}

\section{Lecture 4}
\subsection{Witt equivalence and the Witt ring}
Throughout this subsection, all quadratic spaces are consider over a fixed field $K$, and we assume $\charac(K) \neq 2$.
\begin{defi}
Let $(V^{(1)}, q^{(1)})$ and $(V^{(2)}, q^{(2)})$ be quadratic spaces.
In view of \Cref{M-T:Witt-Decomposition} we may write
\begin{align*}
(V^{(1)}, q^{(1)}) &\cong (V_t^{(1)}, q_t^{(1)}) \perp (V_h^{(1)}, q_h^{(1)}) \perp (V_a^{(1)}, q_a^{(1)}) \\
(V^{(2)}, q^{(2)}) &\cong (V_t^{(2)}, q_t^{(2)}) \perp (V_h^{(2)}, q_h^{(2)}) \perp (V_a^{(2)}, q_a^{(2)})
\end{align*}
where
\begin{itemize}
\item $(V_t^{(1)}, q_t^{(1)})$ and $(V_t^{(2)}, q_t^{(2)})$ are totally isotropic,
\item $(V_h^{(1)}, q_h^{(1)})$ and $(V_h^{(2)}, q_h^{(2)})$ are hyperbolic (or zero),
\item $(V_a^{(1)}, q_a^{(1)})$ and $(V_a^{(2)}, q_a^{(2)})$ are anisotropic.
\end{itemize}
We say that $(V^{(1)}, q^{(1)})$ and $(V^{(2)}, q^{(2)})$ are \emph{Witt equivalent}\index{Witt equivalent} if $\dim V_t^{(1)}  =\dim V_t^{(2)}$ and $(V_a^{(1)}, q_a^{(1)}) \cong (V_a^{(2)}, q_a^{(2)})$.
We denote this by $(V^{(1)}, q^{(1)}) \equiv (V^{(2)}, q^{(2)})$.
\end{defi}
\Cref{M-T:Witt-Decomposition} yields that this is indeed a well-defined equivalence relation on the class of quadratic spaces over $K$.
One has the following easy observations.
\begin{prop}
Let $(V_1, q_1)$ and $(V_2, q_2)$ be quadratic spaces.
\begin{enumerate}
\item $(V_1, q_1) \cong (V_2, q_2)$ if and only if $(V_1, q_1) \equiv (V_2, q_2)$ and $\dim V_1 = \dim V_2$.
\item In every Witt equivalence class, there is up to isometry a unique anisotropic quadratic space.
In particular, if $(V_1, q_1) \equiv (V_2, q_2)$ and both are anisotropic, then $(V_1, q_1) \cong (V_2, q_2)$.
\end{enumerate}
\end{prop}

For a quadratic space $(V, q)$, let us denote by $[(V, q)]$ its Witt equivalence class.
Let us denote by $W(K)$ the set of equivalence classes of nonsingular quadratic spaces up to Witt equivalence.
We will see now that this set can naturally be given a ring structure.
\begin{thm}\label{T:WittRing}
The rules
\begin{align*}
&\perp : W(K) \times W(K) \to W(K) : ([(V_1, q_1)], [(V_2, q_2)]) \to [(V_1 \times V_2, q_1 \perp q_2)] \text{ and}\\
&\otimes : W(K) \times W(K) \to W(K) : ([(V_1, q_1)], [(V_2, q_2)]) \to [(V_1 \otimes V_2, q_1 \otimes q_2)]
\end{align*}
are well-defined binary operations on $W(K)$, making $W(K)$ into a commutative ring with addition $\perp$ and multiplication $\otimes$.
The class of the zero-dimensional form $[\langle \rangle_K]$ is a neutral element for $\perp$, and $[\langle 1 \rangle_K]$ is a neutral element for $\otimes$.
Given $[(V, q)] \in W(K)$, its additive inverse is given by $[(V, -q)]$.
\end{thm}
\begin{proof}
We first prove the well-definedness.
That is, assume $(V_1, q_1), (V_1', q_1'), (V_2, q_2), (V_2, q_2')$ are such that $(V_1, q_1) \equiv (V_1', q_1')$ and $(V_2, q_2) \equiv (V_2', q_2')$, we need to show that $(V_1 \times V_2, q_1 \perp q_2) \equiv (V_1' \times V_2', q_1' \perp q_2')$ and $(V_1 \otimes V_2, q_1 \otimes q_2) \equiv (V_1' \otimes V_2', q_1' \otimes q_2')$.
Since nonsingular quadratic spaces are Witt equivalent if and only if they are isometric after adding a number of copies of the hyperbolic plane to one of them, it suffices to consider the case $(V_1, q_1) = (V_1', q_1')$ and $(V_2', q_2') = (V_2, q_2) \perp \mbb{H}_K$.

We compute that
\begin{displaymath}
(V_1, q_1) \perp ((V_2, q_2) \perp \mbb{H}_K) \cong ((V_1, q_1) \perp (V_2, q_2)) \perp \mbb{H}_K \equiv (V_1, q_1) \perp (V_2, q_2)
\end{displaymath}
as desired.
Similarly
\begin{align*}
(V_1, q_1) \otimes ((V_2, q_2) \perp \mbb{H}_K) &\cong (V_1, q_1) \otimes (V_2, q_2) \perp (V_1, q_1) \otimes \mbb{H}_K \\
&\cong (V_1, q_1) \otimes (V_2, q_2) \perp \dim(V_1) \times \mbb{H}_K \\
&\equiv (V_1, q_1) \otimes (V_2, q_2)
\end{align*}
where the second isometry follows from \Cref{M-C:tensor-product-hyperbolic}.
This shows that the operations $\perp$ and $\otimes$ are well-defined on $W(K) \times W(K)$.
The associativity, commutativity and distributivity are immediate from the corresponding properties for $\perp$ and $\otimes$ on quadratic spaces.
That $[\langle \rangle_K]$ is a neutral element for $\perp$ and $[\langle 1 \rangle_K]$ is a neutral element for $\otimes$, is readily verified.
Finally, that $[(V, -q)] = -[(V, q)]$ is a reformulation of \Cref{M-P:q-q-hyperbolic}.
\end{proof}
\begin{defi}
The set $W(K)$ endowed with the ring structure described in \Cref{T:WittRing} is called the \emph{Witt ring of $K$}\index{Witt ring}.
\end{defi}
\begin{prop}\label{P:fundamental-ideal}
$W(K)$ has a unique ideal of index $2$, which is given by
$$ I(K) = \lbrace [(V, q)] \mid \dim V \text{ even} \rbrace.$$
\end{prop}
\begin{proof}
Observe that, if two nonsingular quadratic spaces are Witt equivalent, then their dimensions differ by an even number.
In particular, if one of them has even dimension, then the other too.
It is easy to see that $I(K)$ is an ideal.
Furthermore, it has index 2, because for any quadratic space $(V, q)$, either $[(V, q)] \in I(K)$, or $[(V, q) \perp \langle 1 \rangle_K] \in I(K)$.

Assume that $J$ is another ideal of $W(K)$ of index $2$.
For $a, b \in K^\times$, we have that $[\langle a \rangle_K], [\langle b \rangle_K] \in W(K)^\times \subseteq W(K) \setminus J$, hence $[\langle a, b \rangle_K] \in J$.
In view of \Cref{M-C:diagonalisation}, we conclude that $J$ contains all classes of quadratic spaces of even dimension, hence $I(K) \subseteq J$.
But then $I(K) = J$.
\end{proof}
\begin{defi}
The ideal $I(K)$ described in \Cref{P:fundamental-ideal} is called the \emph{fundamental ideal of $W(K)$}\index{fundamental ideal}.
\end{defi}

\begin{rem}
Over a field $K$ with $\charac(K) = 2$, the situation is more subtle. There are natural operations $\perp$ and $\otimes$ on the class of \textit{symmetric bilinear spaces} over $K$, and this allows one to define a Witt ring $W(K)$ of nonsingular symmetric bilinear forms.
On the class of quadratic spaces over $K$ there is no natural notion of tensor product, but one can still define a group operation $\perp$, and one obtains a different object from $W(K)$: the quadratic Witt group $I_q(K)$.
While $I_q(K)$ is not a ring, it does carry an action by $W(K)$: $I_q(K)$ is a $W(K)$-module.
See \autocite[Sections 2, 8]{ElmanKarpenkoMerkurjev} for more on this.
\end{rem}

\subsection{Determinants and discriminants}
We briefly introduce the concept of the determinant of a symmetric bilinear form.
This allows us to simplify certain computations with small-dimensional quadratic forms.
\begin{prop}
Let $(V_1, B_1)$ and $(V_2, B_2)$ be isometric symmetric bilinear spaces with bases $\mf{B}_1$ and $\mf{B}_2$.
Then $\det (M_\mf{B_1}(B_1)) \equiv \det (M_{\mf{B}_2}(B_2)) \bmod K^{\times 2}$.
\end{prop}
\begin{proof}
It suffices to consider the case $V_1 = V_2 = K^n$ for $n = \dim(V_1)$, and where $\mf{B}_1$ is the canonical basis $\lbrace e_1, \ldots, e_n \rbrace$.
Let $C \in \mbb{M}_n(K)^\times$ be the base change matrix between $\mf{B}_1$ and $\mf{B}_2$, i.e. ~such that $\mf{B}_2 = \lbrace Ce_1, \ldots, Ce_n \rbrace$.
We see that for column vectors $v, w \in K^n$ we have
\begin{displaymath}
v^T C^T M_{\mf{B}_{1}}(B) C w = B(Cv, Cw) = v^T M_{\mf{B}_2}(B) w
\end{displaymath}
whence $M_{\mf{B}_2}(B) = C^T M_{\mf{B}_{1}}(B) C$ and hence $\det(M_{\mf{B}_2}(B)) = \det(M_{\mf{B}_{1}}(B)) \det(C)^2 \equiv \det(M_{\mf{B}_2}(B)) \bmod K^{\times 2}$ as desired.
\end{proof}
\begin{defi}
For a nonsingular symmetric bilinear space $(V, B)$, we define the \emph{determinant}\index{determinant} of $(V, B)$ (or simply of $B$) to be the equivalence class of $\det(M_{\mf{B}}(B))$ in $K^\times/K^{\times 2}$, where $\mf{B}$ is any basis of $V$.
We denote it simply by $\det(V, B)$.

If $\charac(K) \neq 2$ and $(V, q)$ is a quadratic space over $K$, we define its determinant as the determinant of $(V, \frac{\mf{b}_q}{2})$.
\end{defi}
For the rest of this subsection, assume that all quadratic spaces are considered over a field $K$ with $\charac(K) \neq 2$.

\begin{prop}\label{P:det-computation-rules}
We have the following properties.
\begin{enumerate}
\item For nonsingular quadratic spaces $(V_1, q_1)$ and $(V_2, q_2)$ we have $\det ((V_1, q_1) \perp (V_2, q_2)) = \det(V_1, q_1) \cdot \det(V_2, q_2)$.
\item For $a_1, \ldots, a_n \in K^\times$ we have $\det(\langle a_1, \ldots, a_n \rangle_K) \equiv a_1 \cdots a_n \bmod K^{\times 2}$.
\item $\det(\mbb{H}_K) \equiv -1 \bmod K^{\times 2}$.
\end{enumerate}
\end{prop}
\begin{proof}
These can be verified easily via the definition.
\end{proof}

As announced, determinants are a useful invariant of quadratic spaces which can help to simplify certain calculations.
We give an important example.
\begin{prop}\label{P:binary-form-determinant}
Let $a, b, c \in K^\times$ and assume that $c \in D_K(\langle a, b \rangle_K)$.
Then $\langle a, b \rangle_K \cong \langle c, abc \rangle_K$.
\end{prop}
\begin{proof}
By \Cref{M-P:diagonalisation} we have $\langle a, b \rangle_K \cong \langle c, d \rangle_K$ for some $d \in K^\times$.
But since $cd \equiv \det(\langle c, d \rangle_K) \equiv \det(\langle a, b \rangle_K) \equiv ab \bmod K^\times$, we must have $d \equiv abc \bmod K^{\times 2}$, whereby $\langle c, d \rangle_K \cong \langle c, abc \rangle_K$.
This concludes the proof.
\end{proof}

\subsection{Multiplicative forms}
When $(V, q)$ is a quadratic space, the set $D_K(q)$ of elements of $K^\times$ represented by $q$ is in general just a subset of $K^\times$.
We now consider a class of quadratic forms where this is in fact a subgroup.
\begin{defi}
Let $(V, q)$ be a quadratic space.
We call the set
$$ G_K(q) = \lbrace a \in K^\times \mid (V, q) \cong (V, aq) \rbrace $$
the set of \emph{similarity factors of $(V, q)$}\index{similarity factor}.

By a \emph{multiplicative form over $K$}\index{multiplicative form} (some books use the term \emph{round form}) we mean a nonsingular quadratic form $q$ for which $D_K(q) = G_K(q)$.
\end{defi}
\begin{eg}
Every hyperbolic form is multiplicative, see \Cref{M-C:hyperbolic-form}.
\end{eg}
\begin{prop}\label{P:GKq-properties}
Let $(V, q)$ be a nonsingular quadratic space over $K$.
\begin{enumerate}
\item $G_K(q)$ is a subgroup of $K^\times$ that contains $K^{\times 2}$.
\item $G_K(q) \cdot D_K(q) = D_K(q)$.
\end{enumerate}
\end{prop}
\begin{proof}
The first part is clear.
For the second part, consider $a \in G_K(q)$ and $d \in D_K(q)$, then $ad \in D_K(aq) = D_K(q)$.
\end{proof}
For the rest of this subsection, assume $\charac(K) \neq 2$.
\begin{thm}[Witt]\label{T:Witt-multiplicative-forms}
Let $q$ be a multiplicative form over $K$ and $a \in K^\times$.
Then the form $\langle 1, a \rangle_K \otimes q$ is multiplicative.
Moreover, if $q$ is anisotropic, then $\langle 1, a \rangle_K \otimes q$ is either anisotropic or hyperbolic.
\end{thm}
\begin{proof}
Let $q' = \langle 1, a \rangle_K \otimes q$. We have $1 \in G_K(q) = D_K(q) \subseteq D_K(q')$ and hence $G_K(q') \subseteq D_K(q')$ by \Cref{P:GKq-properties}.
Further, observe that $D_K(q) \cup aD_K(q) = G_K(q) \cup aG_K(q) \subseteq G_K(q')$. 
Now consider $c \in D_K(q') \setminus (D_K(q) \cup aD_K(q))$ arbitrary.
Then there exist $s, t \in D_K(q) = G_K(q)$ such that $c \in D_K(\langle s, at \rangle_K)$.
By \Cref{P:binary-form-determinant} it follows that $\langle s, at \rangle_K \cong \langle c, acst \rangle_K$.
We now compute that
\begin{align*}
q' &\cong q \perp a q \cong sq \perp atq \cong \langle s, at \rangle_K \otimes q \cong \langle c, acst \rangle_K \otimes q \\
&\cong cq \perp acstq \cong cq \perp acq \cong cq'
\end{align*}
whereby $c \in G_K(q')$.
Since $c \in D_K(q')$ was chosen arbitrarily, we conclude that $q'$ is multiplicative.

For the second part, assume that $q$ is anisotropic and $q'$ is isotropic.
Then there exist $s, t \in D_K(q) = G_K(q)$ with $\langle s, at \rangle_K \cong \mbb{H}_K$.
We compute that
$$ q' \cong q \perp aq \cong sq \perp atq \cong \langle s, at \rangle_K \otimes q \cong \mbb{H}_K \otimes q $$
which is hyperbolic by \Cref{M-C:tensor-product-hyperbolic}.
\end{proof}
\begin{defi}
For $n \in \nat$ and $a_1, \ldots, a_n \in K^\times$, we use the notation
$$ \llangle a_1, \ldots, a_n \rrangle_K = \langle 1, -a_1 \rangle \otimes \ldots \otimes \langle 1, -a_n \rangle_K. $$
In particular, $\llangle \rrangle_K = \langle 1 \rangle_K$, and $\llangle a_1 \rrangle_K = \langle 1, -a_1 \rangle_K$.
We call a form which is isometric to $\llangle a_1, \ldots, a_n \rrangle_K$ for some $a_1, \ldots, a_n \in K^\times$ an \emph{$n$-fold Pfister form}\index{Pfister form}.
\end{defi}
\begin{thm}[Pfister]\label{T:Pfister-forms}
Let $q$ be a Pfister form over $K$.
Then $q$ is multiplicative, and either anisotropic or hyperbolic.
\end{thm}
\begin{proof}
Assume that $q$ is an $n$-fold Pfister form; we proceed by induction on $n$.
For $n = 0$ we have $q \cong \langle 1 \rangle_K$; this form is anisotropic and $D_K(q) = K^{\times 2} = G_K(q)$.
Assume now $n > 0$.
We have that $q \cong \langle 1, -a \rangle_K \otimes q'$ for some $(n-1)$-fold Pfister form $q'$ over $K$.
If $q'$ is anisotropic, then by induction hypothesis, $q'$ is multiplicative, and by \Cref{T:Witt-multiplicative-forms} also $q$ is multiplicative and either anisotropic or hyperbolic.
If $q'$ is isotropic, then by induction hypothesis it is hyperbolic, and then also $q$ is hyperbolic by \Cref{M-C:tensor-product-hyperbolic}.
\end{proof}
We mention the following partial converse to \Cref{T:Pfister-forms}, the proof of which is outside the scope of this course.
We will not use this result in the sequel.
For a quadratic form $q$ over $K$ and a field extension $L/K$, we denote by $q_L$ the quadratic form over $L$ obtained by extending scalars from $K$ to $L$ (we will see a formal definition later, see \Cref{M-D:scalar-extension}).
\begin{thm}[Pfister]\label{T:Pfister-characterisation}
Let $q$ be an anisotropic quadratic form over $K$.
The following are equivalent.
\begin{enumerate}
\item $q$ is a Pfister form,
\item $D_L(q_L)$ is a subgroup of $L^\times$ for every field extension $L/K$,
\item $1 \in D_K(q)$ and for every field extension $L/K$ we have that $q_L$ is either anisotropic or hyperbolic.
\end{enumerate}
\end{thm}
\begin{proof}
See \autocite[Theorem 23.2 and Corollary 23.4]{ElmanKarpenkoMerkurjev}.
\end{proof}
\begin{rem}
Over a field $K$ of characteristic $2$, one can define a notion of Pfister form both for bilinear forms and for quadratic forms.
As usual, we refer to \autocite[Sections 7 and 9]{ElmanKarpenkoMerkurjev} for a characteristic-free exposition.
An example of a $1$-fold quadratic Pfister form is given by $X^2 + XY + aY^2$ for $a \in K$.
These quadratic Pfister forms still satisfy the properties of \Cref{T:Pfister-forms} in characteristic $2$.
\end{rem}

\subsection{Exercises}\label{Exercises-L4}
In all exercises, assume $K$ is a field with $\charac(K) \neq 2$.
\begin{enumerate}
\item Compute the Witt ring of $\cc$ and $\rr$.
\item Let $K$ be finite.
Show the following:
\begin{enumerate}
\item $\lvert K^\times / K^{\times 2} \rvert = 2$,
\item Every nonsingular $2$-dimensional quadratic form over $K$ is universal.
\item Assume $d \in K^\times \setminus K^{\times 2}$.
Every anisotropic quadratic form over $K$ is isometric to precisely one of the following forms:
\begin{displaymath}
\langle \rangle_K \qquad \langle 1 \rangle_K \qquad \langle d \rangle_K \qquad \langle 1, -d \rangle_K
\end{displaymath}
\item If $\lvert K \rvert \equiv 1 \bmod 4$, then $-1 \in K^{\times 2}$ and $WK \cong (\zz/2\zz)[T]/(T^2 + 1)$.
\item If $\lvert K \rvert \equiv 3 \bmod 4$, then $-1 \not\in K^{\times 2}$ and $WK \cong \zz/4\zz$.
\end{enumerate}
\item
Show that for $a, b \in K^\times$ and a Pfister form $q$ over $K$ we have $\llangle a \rrangle_K \otimes q \cong \llangle b \rrangle_K \otimes q$ if and only if $ab \in D_K(q)$.
\item Show that for $a, b, c, d \in K^\times$ we have $\llangle a, b \rrangle_K \cong \llangle c, d \rrangle_K$ if and only if there exists $e \in K^\times$ with
$$ \llangle a, b \rrangle_K \cong \llangle a, e \rrangle_K \cong \llangle c, e \rrangle_K \cong \llangle c, d \rrangle_K.$$
\item\label{ex:4-dim-Pfister} Let $q$ be a $4$-dimensional nonsingular quadratic form over $K$ with $1 \in D_K(q)$ and $\det(q) \equiv 1 \bmod K^{\times 2}$.
Show that $q$ is a Pfister form.
\item Let $q$ be a universal $3$-dimensional quadratic form over $K$.
Show that $q$ is isotropic.
\item Show that $D_\qq(\langle 1, 1 \rangle_\qq)$ is a subgroup of $\qq^\times$.
Is the same true for $D_\qq(\langle 1, 1, 1 \rangle_\qq)$?
\item Give an example of an anisotropic quadratic form which is multiplicative but not a Pfister form.
\item Let $n \in \nat$ and suppose that $-1$ is a sum of $2^{n+1} - 1$ squares in $K$.
Show that $-1$ is a sum of $2^n$ squares in $K$.
\end{enumerate}
\end{document}