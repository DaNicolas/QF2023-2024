\documentclass[12pt, leqno, british]{amsart}
\usepackage[style=alphabetic, backend=biber]{biblatex}
\usepackage{a4, amsmath}
\usepackage{mathtools}
\usepackage{amssymb}
\usepackage{amsthm, amscd, mathdots}
\swapnumbers
\usepackage{enumerate}
\usepackage{hyperref}
\usepackage{cleveref}
\usepackage{csquotes}
\usepackage{color}
\usepackage{datetime}
\usepackage{xr, standalone, import}

\theoremstyle{definition}
\newtheorem{defi}{Definition}[subsection]
\theoremstyle{plain}
\newtheorem{prop}[defi]{Proposition}
\newtheorem{lem}[defi]{Lemma}
\newtheorem{thm}[defi]{Theorem}
\newtheorem{cor}[defi]{Corollary}
\newtheorem{ques}[defi]{Question}
\theoremstyle{remark}
\newtheorem{rem}[defi]{Remark}
\newtheorem{eg}[defi]{Example}
\newtheorem{egs}[defi]{Examples}

\newcommand{\mc}{\mathcal}
\newcommand{\mf}{\mathfrak}
\newcommand{\mbb}{\mathbb}
\newcommand{\nat}{\mbb N}
\newcommand{\cc}{\mathbb C}
\newcommand{\rr}{\mathbb R}
\newcommand{\qq}{\mbb Q}
\newcommand{\ovl}{\overline}
\newcommand{\ff}{\mbb F}
\newcommand{\zz}{\mbb Z}

\DeclareMathOperator{\charac}{char}
\DeclareMathOperator{\id}{id}
\DeclareMathOperator{\Frac}{Frac}
\DeclareMathOperator{\Ker}{Ker}
\DeclareMathOperator{\Img}{Im}
\DeclareMathOperator{\Trd}{Trd}
\DeclareMathOperator{\Tr}{Tr}
\DeclareMathOperator{\Nrd}{Nrd}
\DeclareMathOperator{\GL}{GL}
\DeclareMathOperator{\Gal}{Gal}
\DeclareMathOperator{\ord}{ord}
\DeclareMathOperator{\trdeg}{trdeg}
\DeclareMathOperator{\supp}{supp}
\DeclareMathOperator{\rad}{rad}
\DeclareMathOperator{\sign}{sign}
\newcommand{\disc}{\mathrm{d}}

\newcommand{\llangle}{\langle\!\langle}
\newcommand{\rrangle}{\rangle\!\rangle}
\addbibresource{../bibliography.bib}
\externaldocument[M-]{../Lecture-notes}

\author{Nicolas Daans}
\address{Charles University, Faculty of Mathematics and Physics, Department of Algebra, Sokolov\-sk\' a 83, 18600 Praha~8, Czech Republic.}
\email{nicolas.daans@matfyz.cuni.cz}

\begin{document}

\section{Lecture 5}

\subsection{Powers of the fundamental ideal}
Assume throughout that $K$ is a field with $\charac(K) \neq 2$ and that all quadratic spaces are considered over $K$.

We will consider powers of the fundamental ideal $IK$ of the Witt ring $WK$. For a natural number $n$, we denote by $I^n K$ the ideal of $WK$ generated by products of $n$ elements in $IK$.
By convention, we set $I^0K = WK$.
We obtain a natural filtration
$$ WK = I^0K \supseteq I^1 K = IK \supseteq I^2 K \supseteq I^3 K \supseteq \ldots $$
We can try to understand the group $WK$ better by studying the ideals $I^n K$, and/or by studying the quotients $I^n K / I^{n+1} K$.
We already know that $WK/I^1K \cong \zz / 2\zz$, see \Cref{M-P:fundamental-ideal}.
\begin{prop}\label{P:generators-InK}
For $n \geq 1$, the ideal $I^n K$ is generated as a group by the Witt classes of $n$-fold Pfister forms in $K$.
\end{prop}
\begin{proof}
First observe that, for $a, b \in K^\times$, we have
$$ \langle a, b \rangle_K \equiv \langle a, b \rangle_K \perp \mbb{H}_K \cong \langle 1, a \rangle_K \perp -\langle 1, -b \rangle_K \cong \llangle -a \rrangle_K \perp -\llangle b \rrangle_K.$$
Since every nonsingular binary quadratic form is isometric to $\langle a, b \rangle_K$ for some $a, b \in K^\times$ and since binary quadratic forms generate $IK$, we conclude that $IK$ is generated as a group by $1$-fold Pfister forms.
Since an $n$-fold Pfister form is by definition a product of $n$ $1$-fold Pfister forms, we conclude that $I^n K$ is generated by $n$-fold Pfister forms, as desired.
\end{proof}
A quadratic form over $K$ which is isometric to $a\pi$ for a Pfister form $\pi$ and an element $a \in K^\times$ is called a \emph{scaled Pfister form}\index{Pfister form!scaled}.
Observe that the class of a scaled $n$-fold Pfister form lies in $I^nK$.

Our first goal will be to understand the quotient $I K/I^2K$.
\begin{lem}\label{L:split-off-2fold-Pfister}
Let $(V, q)$ a nonsingular quadratic space over $K$, assume $\dim(V) \geq 3$.
There exists a quadratic space $(W, q')$ with $\dim(W) = \dim(V) - 2$ and a scaled $2$-fold Pfister form $(P, q_P)$ such that $(V, q) \equiv (W', q') \perp (P, q_P)$.
\end{lem}
\begin{proof}
In view of \Cref{M-C:diagonalisation} it suffice to consider the case where $(V, q) = \langle a, b, c \rangle_K$ for $a, b, c \in K^\times$.
Now set $q' = \langle -abc \rangle_K$ and $q_P = abc\llangle -ab, ac \rrangle_K$.
We have $\dim(q') = 1$ and we compute that
$$q' \perp q_P \cong \langle -abc , abc \rangle_K \perp \langle a, b, c \rangle_K \cong \mbb{H}_K \perp q \equiv q.$$
Hence $q'$ and $q_P$ are as desired.
\end{proof}
\begin{defi}
Let $(V, q)$ be a nonsingular quadratic space.
We define its \emph{discriminant}\index{discriminant} (in some books called \emph{signed determinant}) to be
$$ \disc(V, q) = (-1)^{\binom{\dim(V)}{2}} \det(V, q) \in K^\times/K^{\times 2}.$$
\end{defi}
Observe that for a natural number $n$ we have
\begin{equation*}\label{E:binom}
(-1)^{\binom{n}{2}} = \begin{cases}
1 &\text{ if } n \equiv 0, 1 \bmod 4 \\
-1 &\text{ if } n \equiv 2, 3 \bmod 4
\end{cases}.
\end{equation*}
In particular, if $m$ and $n$ are two natural numbers and at least one of them is even, then it follows that
\begin{equation}\label{E:binom2}
(-1)^{\binom{m}{2}}(-1)^{\binom{n}{2}} = (-1)^{\binom{m+n}{2}}.
\end{equation}
\begin{prop}\label{P:disc-map}
If $(V, q)$ and $(V', q')$ are Witt equivalent nonsingular quadratic spaces, then $\disc(V, q) = \disc(V', q')$.
Furthermore, the map
$$ IK \to K^\times/K^{\times 2} : [(V, q)] \mapsto \disc(V, q) $$
is a well-defined surjective group homomorphism with kernel $I^2K$.
In particular, $IK/I^2K \cong K^\times / K^{\times 2}$.
\end{prop}
\begin{proof}
For the first part, we need to check that if $(V, q) \equiv (V', q')$, then $\disc(V, q) = \disc(V', q')$.
It suffices to consider the case where $(V', q') = (V, q) \perp \mbb{H}_K$.
We compute using \Cref{M-P:det-computation-rules} and \cref{E:binom2} that
\begin{align*}
\disc((V, q) \perp \mbb{H}_K) &= (-1)^{\binom{\dim(V)+2}{2}}\det((V, q) \perp \mbb{H}_K ) \\
&= - (-1)^{\binom{\dim(V)}{2}} \det(V, q) \det(\mbb{H}_K) \\
&= (-1)^{\binom{\dim(V)}{2}} \det(V, q) = d(V, q)
\end{align*}
as desired.
This also shows that the given map is well-defined.

The fact that it is a group homomorphism is now also immediate from \Cref{M-P:det-computation-rules} and \cref{E:binom2}.
For the surjectivity, it suffices to observe that $\disc(\langle 1, -a \rangle_K) \equiv a \bmod K^{\times 2}$ for $a \in K^\times$.

We now compute the kernel of the morphism.
One computes that, for any $a, b \in K^\times$, we have
$$ \disc(\llangle a, b \rrangle_K) = \disc(\langle 1, -a, -b, ab \rangle_K) \equiv 1 \bmod K^{\times 2}.$$
So, any equivalence class of a $2$-fold Pfister form lies in the kernel of $\disc$.
In view of \Cref{P:generators-InK} we conclude that $I^2K \subseteq \Ker(\disc)$.

For the converse implication, consider $\zeta \in \Ker(\disc)$.
By \Cref{L:split-off-2fold-Pfister} we have $\zeta \equiv [(V, q)] \bmod I^2K$ where $\dim(V) = 2$.
Since $I^2K \subseteq \Ker(\disc)$ by the previous paragraph, we conclude that $\disc(V, q) = \disc(\zeta) \equiv 1 \bmod K^{\times 2}$.
But then $\det(V, q) = -1$, which implies $(V, q) \cong \mbb{H}_K$, whereby $[(V, q)] = 0$, and we conclude that $\zeta \in I^2K$ as desired.
\end{proof}

\begin{prop}\label{P:boundI2}
Let $(V, q)$ be a nonsingular quadratic space with $[(V, q)] \in I^2K$ and $m = \dim(V)/2 - 1$.
There exist scaled $2$-fold Pfister forms $\pi_1, \ldots, \pi_m$ such that
$ [(V, q)] = \sum_{i=1}^m [\pi_i] $.
\end{prop}
\begin{proof}
If $m = 0$ then, as in the proof of \Cref{P:disc-map}, we see that $(V, q)$ must be hyperbolic, hence $[(V, q)] = 0$.
The general case now follows from \Cref{L:split-off-2fold-Pfister} by induction on $m$.
\end{proof}
\begin{ques}\label{Q:boundIn}
Let $n, d \in \nat^+$.
Does there exist a natural number $m$ such that every $d$-dimensional quadratic space $(V, q)$ with $[(V, q)] \in I^n K$ is Witt equivalent to a sum of $m$ scaled Pfister forms?
\end{ques}
For $n = 1$ the answer is easy (every binary nonsingular quadratic form is a scaled Pfister form, so one can take $m = d/2$), and for $n = 2$ one can take $m = d/2 - 1$ by \Cref{P:boundI2}.
For $n = 3$ it is known that such a number $m$ exists, and that it grows at least exponentially as a function of $d$ \autocite{EssentialDimensionI3K}.
For $n > 3$ it is completely open whether such a number $m$ exists in general.
Of course, over many specific fields $K$, often the situation is much easier.

We mention the following major theorem, without providing a proof.
\begin{thm}[Arason-Pfister Hauptsatz, 1971]\label{T:Arason-Pfister}
Let $n \in \nat$ and let $(V, q)$ be a nonsingular quadratic space with $[(V, q)] \in I^n K$.
\begin{enumerate}
\item Either $\dim(V) \geq 2^n$ or $(V, q)$ is hyperbolic.
\item If $\dim(V) = 2^n$, then $(V, q)$ is a scaled $n$-fold Pfister form.
\end{enumerate}
\end{thm}
\begin{proof}
The first part is \autocite[Theorem 23.7]{ElmanKarpenkoMerkurjev}.
The second part follows from combining the first part with \Cref{M-T:Pfister-characterisation}.
\end{proof}
\begin{cor}
We have $\bigcap_{n \in \nat} I^n K = \lbrace 0 \rbrace$.
\end{cor}
\begin{proof}
Consider a non-zero element of $WK$, then it is of the form $[(V, q)]$ for some non-zero anisotropic quadratic form $q$.
For $n > \log_2(\dim(V))$ we have $[(V, q)] \not\in I^n K$ by \Cref{T:Arason-Pfister}.
\end{proof}

\subsection{Symbols in $I^nK/I^{n+1}K$}
Assume throughout that $K$ is a field with $\charac(K) \neq 2$.
We want to give a description of the quotients $I^nK/I^{n+1}K$ through \emph{generators and relations}.
We know from \Cref{P:generators-InK} that the group $I^nK$ is generated by the classes of $n$-fold Pfister forms over $K$, hence the same holds for $I^{n+1}K$.
We also know that $WK/IK \cong \zz/2\zz$, and that $IK/I^2K \cong K^\times/K^{\times 2}$ via the signed discriminant map (see \Cref{P:disc-map}); its inverse is the map $K^\times/K^{\times 2} \to IK/I^2K$ mapping the class of an element $a \in K^\times$ to the quadratic form $\langle 1, -a \rangle_K$.
We now seek to generalise this to higher powers of the fundamental ideal.
The presentation in this subsection gives an ad hoc introduction to a branch of mathematics closely related to quadratic form theory, called \emph{Algebraic $K$-Theory}, or sometimes \emph{Milnor's $K$-Theory}\index{Milnor's $K$-Theory}.
\begin{defi}
For $n \in \nat$ and $a_1, \ldots, a_n \in K^\times$, we denote by $\lbrace a_1, \ldots, a_n \rbrace_K$ the equivalence class of $\llangle a_1, \ldots, a_n \rrangle_K$ in $I^nK/I^{n+1}K$.
We call such elements $\lbrace a_1, \ldots, a_n \rbrace_K \in I^nK/I^{n+1}K$ \emph{symbols}\index{symbol}.
\end{defi}
We have a map $(K^\times)^n \to I^nK/I^{n+1}K : (a_1, \ldots, a_n) \mapsto \lbrace a_1, \ldots, a_n \rbrace_K$.
Let us phrase some basic properties of this map.
%We denote the operation on $I^nK/I^{n+1}K$ induced by the orthogonal sum $\perp$ by $+$.
\begin{prop}\label{P:K-theory-axioms}
We have the following for $a_1, \ldots, a_n \in K^\times$,
\begin{enumerate}
\item\label{it:multilinearity} \textit{(multilinearity)} For $i \in \lbrace 1, \ldots, n \rbrace$ and $a_i' \in K^\times$ $$ \lbrace a_1, \ldots, a_{i}a_i', \ldots a_n \rbrace_K = \lbrace a_1, \ldots, a_i, \ldots, a_n \rbrace_K + \lbrace a_1, \ldots, a_i', \ldots, a_n \rbrace_K, $$
\item\label{it:2-torsion} \textit{($2$-torsion)}
$$ 2\times \lbrace a_1,\ldots, a_n \rbrace_K = \lbrace a_1,\ldots, a_n \rbrace_K + \lbrace a_1,\ldots, a_n \rbrace_K = 0,$$
\item\label{it:Steinberg} \textit{(Steinberg relation)} If $i \in \lbrace 1, \ldots, n-1 \rbrace$ is such that $a_i + a_{i+1} = 1$, then $\lbrace a_1, \ldots, a_n \rbrace_K = 0$.
\end{enumerate}
\end{prop}
\begin{proof}
\eqref{it:multilinearity}: 
Consider the $(n-1)$-fold Pfister form $q = \llangle a_1, \ldots, a_{i-1}, a_{i+1}, \ldots, a_n \rrangle$.
We need to show that $[q \otimes \llangle a_i \rrangle_K] + [q \otimes \llangle a_i' \rrangle_K] \equiv [q \otimes \llangle a_ia_i' \rrangle_K] \bmod I^{n+1}K$.
We compute that
\begin{eqnarray*}
&&q \otimes \llangle a_i \rrangle_K \perp q \otimes \llangle a_i' \rrangle_K \perp -q \otimes \llangle a_ia_i' \rrangle_K \\
\cong && (q \perp -a_iq) \perp (q \perp -a_i'q) \perp -(q \perp a_ia_i'q) \\
\cong && (q \perp -a_iq \perp -a_i'q \perp a_ia_i'q) \perp (q \perp -q) \\
\cong &&q \otimes \llangle a_i, a_{i+1} \rrangle_K \perp q \otimes \llangle -1 \rrangle_K.
\end{eqnarray*}
Since $q \otimes \llangle a_i, a_{i+1} \rrangle_K$ is an $(n+1)$-fold Pfister form and $q \otimes \llangle -1 \rrangle$ is hyperbolic (as $\llangle 1 \rrangle_K \cong \mbb{H}_K$, see also \Cref{M-C:tensor-product-hyperbolic})
we obtain that $$[q \otimes \llangle a_i \rrangle_K] + [q \otimes \llangle a_i' \rrangle_K] - [q \otimes \llangle a_ia_i' \rrangle_K] = [q \otimes \llangle a_i, a_{i+1} \rrangle_K \perp q \otimes \llangle -1 \rrangle_K] \in I^{n+1}K.$$
From this, the desired statement follows.

\eqref{it:2-torsion}
Set $q = \llangle a_1, \ldots, a_n \rrangle_K$.
We have $[q] + [q] = [q \perp q] = [q \otimes \llangle -1 \rrangle] \in I^{n+1}K$.
From this the statement follows.

\eqref{it:Steinberg} It suffices to show that $\llangle a_1, \ldots, a_n \rrangle_K$ is hyperbolic whenever $a_i + a_{i+1} = 1$.
In view of \Cref{M-T:Pfister-forms} it even suffices to show that $\llangle a_1, \ldots, a_n \rrangle_K$ is isotropic.
This follows because $\llangle a_1, \ldots, a_n \rrangle_K$ contains as a subform $\langle 1, -a_i, -a_{i+1} \rangle_K$, which is isotropic since $1^2 - a_i1^2 - a_{i+1}1^1 = 0$.
\end{proof}
The following properties can be derived from the properties of Pfister forms and the definition of $I^nK/I^{n+1}K$, but, more interestingly, they can be proved using only only the three properties from \Cref{P:K-theory-axioms} as axioms.
\begin{cor}\label{C:K-theory-axioms}
We have the following for $a_1, \ldots, a_n \in K^\times$:
\begin{enumerate}
\item If $i \in \lbrace 1, \ldots, n-1 \rbrace$ is such that $a_i + a_{i+1} = 0$, then $\lbrace a_1, \ldots, a_n \rbrace_K = 0$.
\item\label{it:permutation-invariance} \textit{(invariance under permutation)} For $i \in \lbrace 1, \ldots, n-1 \rbrace$, $$\lbrace a_1, \ldots, a_i, a_{i+1}, \ldots, a_n \rbrace_K = \lbrace a_1, \ldots, a_{i+1}, a_i, \ldots, a_n \rbrace_K,$$
\item\label{it:scale-square-invariance} For $d \in K^\times$, $\lbrace a_1d^2, a_2, \ldots, a_n \rbrace_K = \lbrace a_1, a_2, \ldots, a_n \rbrace_K$,
\item $\lbrace a_1, \ldots, a_n \rbrace_K = \lbrace a_1 + a_2, -a_1a_2, a_3, \ldots, a_n \rbrace_K$, provided that $a_2 \neq -a_1$.
\end{enumerate}
\end{cor}
\begin{proof}
Exercise.
\end{proof}
In an influential paper from 1970 \cite{Milnor}, John Milnor conjectured that the three properties from \Cref{P:K-theory-axioms} can be used to completely axiomatise the relations between the elements $\lbrace a_1, \ldots, a_n \rbrace_K$.
He was able to establish several special cases (small values of $n$ and specific fields), but the general case was only solved more than thirty years later by the work of Orlov, Vishik, and Voevodsky \cite{OrlovVishikVoevodsky}.
Below is a version of their result, stated in the form of a universal property.
The proof is far beyond the scope of this course, and we will not make use of this result in the course either.
\begin{thm}[Orlov-Vishik-Voevodsky]
Let $n \in \nat$, $G$ an abelian group, and $\Phi : (K^\times)^n \to G$ a map satisfying the properties from \Cref{P:K-theory-axioms}, i.e.~for $a_1, \ldots, a_n \in K^\times$, $i \in \lbrace 1, \ldots, n \rbrace$, $a_i' \in K^\times$,
\begin{itemize}
\item $\Phi(a_1, \ldots, a_ia_i', \ldots, a_n) = \Phi(a_1, \ldots, a_i, \ldots, a_n) + \Phi(a_1, \ldots, a_i', \ldots, a_n)$,
\item $2 \times \Phi(a_1, \ldots, a_n) = 0$,
\item if $a_i + a_{i+1} = 1$, then $\Phi(a_1, \ldots, a_n) = 0$.
\end{itemize}
Then there exists a unique group homomorphism $\tilde{\Phi} : I^nK/I^{n+1}K \to G$ such that $\Phi(a_1, \ldots, a_n) = \tilde{\Phi}(\lbrace a_1, \ldots, a_n \rbrace_K)$ for all $a_1, \ldots, a_n \in K^\times$.
\end{thm}
Finally, we mention the following consequence of the Arason-Pfister Hauptsatz, which can be useful when classifying quadratic forms over a field.
\begin{prop}\label{P:symb-equal}
For $n \in \nat$ and $a_1, \ldots, a_n, b_1, \ldots, b_n \in K^\times$ we have that
$$ \lbrace a_1, \ldots, a_n \rbrace_K = \lbrace b_1, \ldots, b_n \rbrace_K \quad\text{if and only if}\quad \llangle a_1, \ldots, a_n \rrangle_K \cong \llangle a_1, \ldots, a_n \rrangle_K.$$
In particular, we have $\lbrace a_1, \ldots, a_n \rbrace_K = 0$ if and only if $\llangle a_1, \ldots, a_n \rrangle_K$ is hyperbolic.
\end{prop}
\begin{proof}
By definition we have $\lbrace a_1, \ldots, a_n \rbrace_K = \lbrace b_1, \ldots, b_n \rbrace_K$ if and only if $[\llangle a_1, \ldots, a_n \rrangle_K] \equiv [\llangle b_1, \ldots, b_n \rrangle_K] \bmod I^{n+1}K$.
It is thus clear that if $\llangle a_1, \ldots, a_n \rrangle_K \cong \llangle a_1, \ldots, a_n \rrangle_K$, then $\lbrace a_1, \ldots, a_n \rbrace_K = \lbrace b_1, \ldots, b_n \rbrace_K$.

Conversely, suppose $[\llangle a_1, \ldots, a_n \rrangle_K] \equiv [\llangle b_1, \ldots, b_n \rrangle_K] \bmod I^{n+1}K$.
Then the class of $\llangle a_1, \ldots, a_n \rrangle_K \perp -\llangle b_1, \ldots, b_n \rrangle_K$ lies in $I^{n+1}K$ and has dimension $2^{n+1}$, so by \Cref{T:Arason-Pfister} it must be isometric to a scaled $(n+1)$-fold Pfister form.
In particular, by \Cref{M-T:Pfister-forms} it must be either anisotropic or hyperbolic.
It is not anisotropic (since $1 \in D_K(\llangle a_1, \ldots, a_n \rrangle_K) \cap D_K(\llangle b_1, \ldots, b_n \rrangle_K)$, so it must be hyperbolic.
In other words, $[\llangle a_1, \ldots, a_n \rrangle_K \perp -\llangle b_1, \ldots, b_n \rrangle_K] = 0$ in $WK$, whereby $[\llangle a_1, \ldots, a_n \rrangle_K] = [\llangle b_1, \ldots, b_n \rrangle_K]$, and since both forms have the same dimension, we conclude that indeed $\llangle a_1, \ldots, a_n \rrangle_K \cong \llangle b_1, \ldots, b_n \rrangle_K$.

The ``in particular'' part follows by taking $\llangle b_1, \ldots, b_n \rrangle$ hyperbolic (e.g.~setting $b_1 = 1$).
\end{proof}

\subsection{Exercises}
\begin{enumerate}
\item Show the following for a field $K$ with $\charac(K) \neq 2$:
\begin{enumerate}
\item $\lvert K^\times / K^{\times 2} \rvert < \infty$ if and only if, for every $n \in \nat$, there exist up to isomorphism only finitely many anisotropic quadratic forms of dimension $n$, if and only if $WK$ is a noetherian ring,
\item $\lvert WK \rvert < \infty$ if and only if $\lvert K^\times / K^{\times 2} \rvert < \infty$ and $-1$ is a sum of squares in $K$.
\end{enumerate}
\item Give a proof of \Cref{C:K-theory-axioms} using only the computation rules from \Cref{P:K-theory-axioms}.
\end{enumerate}
\end{document}