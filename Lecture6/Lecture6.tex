\documentclass[12pt, leqno, british]{amsart}
\usepackage[style=alphabetic, backend=biber]{biblatex}
\usepackage{a4, amsmath}
\usepackage{mathtools}
\usepackage{amssymb}
\usepackage{amsthm, amscd, mathdots}
\swapnumbers
\usepackage{enumerate}
\usepackage{hyperref}
\usepackage{cleveref}
\usepackage{csquotes}
\usepackage{color}
\usepackage{datetime}
\usepackage{xr, standalone, import}

\theoremstyle{definition}
\newtheorem{defi}{Definition}[subsection]
\theoremstyle{plain}
\newtheorem{prop}[defi]{Proposition}
\newtheorem{lem}[defi]{Lemma}
\newtheorem{thm}[defi]{Theorem}
\newtheorem{cor}[defi]{Corollary}
\newtheorem{ques}[defi]{Question}
\theoremstyle{remark}
\newtheorem{rem}[defi]{Remark}
\newtheorem{eg}[defi]{Example}
\newtheorem{egs}[defi]{Examples}

\newcommand{\mc}{\mathcal}
\newcommand{\mf}{\mathfrak}
\newcommand{\mbb}{\mathbb}
\newcommand{\nat}{\mbb N}
\newcommand{\cc}{\mathbb C}
\newcommand{\rr}{\mathbb R}
\newcommand{\qq}{\mbb Q}
\newcommand{\ovl}{\overline}
\newcommand{\ff}{\mbb F}
\newcommand{\zz}{\mbb Z}

\DeclareMathOperator{\charac}{char}
\DeclareMathOperator{\id}{id}
\DeclareMathOperator{\Frac}{Frac}
\DeclareMathOperator{\Ker}{Ker}
\DeclareMathOperator{\Img}{Im}
\DeclareMathOperator{\Trd}{Trd}
\DeclareMathOperator{\Tr}{Tr}
\DeclareMathOperator{\Nrd}{Nrd}
\DeclareMathOperator{\GL}{GL}
\DeclareMathOperator{\Gal}{Gal}
\DeclareMathOperator{\ord}{ord}
\DeclareMathOperator{\trdeg}{trdeg}
\DeclareMathOperator{\supp}{supp}
\DeclareMathOperator{\rad}{rad}
\DeclareMathOperator{\sign}{sign}
\newcommand{\disc}{\mathrm{d}}

\newcommand{\llangle}{\langle\!\langle}
\newcommand{\rrangle}{\rangle\!\rangle}
\addbibresource{../bibliography.bib}
\externaldocument[M-]{../Lecture-notes}

\author{Nicolas Daans}
\address{Charles University, Faculty of Mathematics and Physics, Department of Algebra, Sokolov\-sk\' a 83, 18600 Praha~8, Czech Republic.}
\email{nicolas.daans@matfyz.cuni.cz}

\begin{document}

\section{Lecture 6}

\subsection{The $p$-adic numbers}
To see that a polynomial $f \in \zz[X_1, \ldots, X_n]$ does not have an integral zero, one can often use modular arithmetic: if $f$ does not have a zero over $\zz/m\zz$ for some $m \in \nat$, then it can certainly not have a zero over $\zz$.
For example, the equation $X^3 + 7XY = 16$ cannot have an integral solution, because modulo $7$ it reduces to $X^3 = 2$, which has no solution in the ring $\zz/7\zz$.

Similarly, modular arithmetic can be used to show that an equation does not have a rational solution.
The most famous example is the equation $X^2 = 2$, which cannot have a solution in $\qq$ by considerations modulo $4$, after writing out a hypothetical solution in $\qq$ as a fraction of two coprime integers.

In this section we shall introduce, for each prime number $p$, a commutative ring $\zz_p$, called the \emph{ring of $p$-adic integers}\index{$p$-adic integers}.
It is an object which shall capture, in a certain sense, all information about solvability of polynomial equations modulo powers of $p$.

\begin{prop}\label{P:Zp-construction}
Let $p \in \mbb P$. Consider the subset of the product ring $A_p = \prod_{n \in \nat^+} \zz/p^n \zz$ given as
$$ \zz_p = \lbrace (x_n + p^n\zz) \in A_p \mid (x_n)_n \in \zz^\nat \text{ s.t. } x_n \equiv x_{n+1} \bmod p^n \text{ for all } n \in \nat^+ \rbrace.$$
We have the following:
\begin{enumerate}[(i)]
\item $\zz_p$ is a subring of $A_p$ of characteristic $0$.
\item $\zz_p$ is an integral domain.
\item The ideal $p\zz_p$ is a maximal ideal of $\zz_p$, and we have for $m \in \nat^+$ that
$$ p^m\zz_p = \lbrace (x_n + p^n\zz)_n \in \zz_p \mid x_m = 0 \rbrace.$$
\item For all $n \in \nat$, $p^n\zz_p \cap \zz = p^n\zz$, and the natural map $\zz \to \zz_p$ induces an isomorphism $\zz/p^n\zz \to \zz_p/p^n\zz_p$.
\end{enumerate}
\end{prop}
\begin{proof}
Exercise.
\end{proof}
\begin{defi}
We call the ring $\zz_p$ constucted in \Cref{P:Zp-construction} the \emph{ring of $p$-adic integers}.
We denote by $\qq_p$ its field of fractions, and call it the \emph{field of $p$-adic numbers}.
\end{defi}
It follows from \Cref{P:Zp-construction} that a polynomial $f \in \zz[X_1, \ldots, X_n]$ which has a root in $\zz_p$ will have a root in $\zz/p^m\zz$ for all $m \in \nat$.
Assuming the Axiom of Choice, the converse statement also holds, see Exercise \eqref{ex:solution-patching}.
We will not need this converse statement in full generality;
instead, let us phrase now a powerful tool to establish explicitly solvability of equations in $\zz_p$; it can be seen as a $p$-adic version of Newton's method from numerical analysis.

For a commutative ring $R$ and a univariate polynomial $f \in R[X]$, we denote by $f' \in R[X]$ its formal derivative\index{formal derivative}, i.e.~if $f = \sum_{i=0}^n a_i X^i$ for $n \in \nat, a_i \in R$, then $f' = \sum_{i=0}^{n-1} (i+1)a_{i+1}X^i$.
\begin{prop}[Hensel's Lemma]\label{P:Hensel}
Let $f \in \zz_p[X]$ and let $x_1 \in \zz$ be such that
$$ f(x_1) \equiv 0 \not\equiv f'(x_1) \bmod p,$$
in other words, $\ovl{x_1}$ is a simple root of $\ovl{f}$ in $\zz/p\zz$.
Then there exists $x \in \zz_p$ with $x - x_1 \in p\zz_p$ and $f(x) = 0$.
\end{prop}
\begin{proof}
It suffices to construct recursively for $n \in \nat^+$ an element $x_{n+1} \in \zz$ with
$$ x_{n+1} \equiv x_n \bmod p^n \text{ and } f(x_{n+1}) \equiv 0 \bmod p^{n+1}.$$
Indeed, we may then set $x = (x_n + p^n\zz)_n$; this is an element of $\zz_p$ by construction, and we further obtain $f(x) = (f(x_n) + p^n\zz)_n = 0$ as desired.

So let $n \in \nat^+$ and assume $x_{n}$ is already given.
We have in particular that
$$ x_{n} \equiv x_1 \bmod p, f(x_n) \equiv 0 \bmod p^n \text{ and } f'(x_n) \equiv f'(x_1) \not\equiv 0 \bmod p. $$
For $e \in \zz$ we have (by a formal version of ``Taylor's Theorem'', see Exercise \eqref{ex:formal-Taylor}):
$$ f(x_n + ep^{n}) \equiv f(x_n) + f'(x_n) ep^{n} \bmod p^{n+1}.$$
Since by assumption $p^{n}$ divides $f(x_n)$ and $f'(x_n) \not\equiv 0 \bmod p$, there exists some $e \in \zz$ with
$$ p^{-n}f(x_n) + f'(x_n)e \equiv 0 \bmod p.$$
Thus, it suffices to set, for this value of $e$, $x_{n+1} = x_n + ep^n$.
\end{proof}
We obtain more properties of the ring $\zz_p$ and the field $\qq_p$.
\begin{prop}\label{P:Zp-properties}
Let $p \in \mbb P$.
We have the following.
\begin{enumerate}[(i)]
\item\label{it:local} $\zz_p$ is a local ring, with unique maximal ideal $p\zz_p$.
In particular, $\zz_p^{\times} = \zz \setminus p\zz_p$.
\item\label{it:Zp-ideal-presentation} Every nonzero ideal of $\zz_p$ is of the form $p^n\zz_p$ for a unique $n \in \nat$; in particular, $\zz_p$ is a principal ideal domain.
\item\label{it:Qp-presentation} Every nonzero element of $\qq_p$ has a unique presentation of the form $p^k u$ for $k \in \zz$ and $u \in \zz_p^\times$.
\end{enumerate}
\end{prop}
\begin{proof}
\eqref{it:local}: We already know from \Cref{P:Zp-construction} that $p\zz_p$ is a maximal ideal of $\zz_p$.
Take $x \in \zz_p \setminus p\zz_p$ and consider the polynomial $f(X) = Xx - 1$.
Since $x \not\in p\zz_p$, its residue $\ovl{x} \in \zz/p\zz \cong \ff_p$ is invertible, so there exists $y_1 \in \zz$ with $f(y) = yx - 1 \equiv 0 \bmod p$.
On the other hand, $f'(y_1) = x \not\equiv 0 \bmod p$.
We conclude by \Cref{P:Hensel} that there exists $y \in \zz_p$ with $0 = f(t) = yx - 1$, i.e.~$y = x^{-1}$.
This shows that $\zz_p \setminus p\zz_p = \zz_p^{\times}$, so $p\zz_p$ is the unique maximal ideal of $\zz_p$, so $\zz_p$ is a local ring.

\eqref{it:Zp-ideal-presentation} and \eqref{it:Qp-presentation}: It suffices to show that every non-zero element of $\zz_p$ has a unique presentation as $p^k u$ for $k \in \nat$ and $u \in \zz_p^\times$, then both statements follow easily.
The uniqueness is clear, as $p\zz_p$ is a prime ideal.
The existence is clear from \eqref{it:local} and the fact that $\bigcap_{n \in \nat} p^n\zz_p = \lbrace 0 \rbrace$ by \Cref{P:Zp-construction}.
\end{proof}
According to part \eqref{it:Qp-presentation} of \Cref{P:Zp-properties}, we can write an arbitrary non-zero element $x \in \qq_p$ as $p^k u$ for $k \in \zz$ and $u \in \zz_p^\times$.
One sees that then $k = \max \lbrace l \in \zz \mid p^{-l} u \in \zz_p \rbrace$.
We can thus define a map
$$ v_p : \qq_p^\times \to \zz : x \mapsto \max \lbrace l \in \zz \mid p^{-l} u \in \zz_p \rbrace. $$
We extend this to a map $\qq_p \to \zz \cup \lbrace \infty \rbrace$ by the convention $v_p(0) = \infty$, and call $v_p$ the \emph{$p$-adic valuation}\index{$p$-adic valuation} on $\qq_p$.
Taking the convention that $a + \infty = \infty$ for all $a \in \zz \cup \lbrace \infty \rbrace$ and that $\infty > a$ for all $a \in \zz$, we obtain the following.
\begin{prop}\label{P:p-adic-valuation}
Let $p \in \mbb P$.
We have
\begin{itemize}
\item $v_p(xy) = v_p(x) + v_p(y)$ for all $x, y \in \qq_p$,
\item $v_p(x + y) \geq \min \lbrace v_p(x), v_p(y) \rbrace$, and equality holds when $v_p(x) \neq v_p(y)$,
\item $\zz_p = \lbrace x \in \qq_p \mid v_p(x) \geq 0 \rbrace$,
\item for $x \in \zz$ we have $v_p(x) = \max \lbrace l \in \nat \mid x \in p^l\zz \rbrace$.
In particular, $\lbrace z \in \qq \mid v_p(z) \geq 0 \rbrace = \zz_p \cap \qq = \lbrace \frac{x}{y} \mid x \in \zz, y \in \zz \setminus p\zz \rbrace$.
\item For $x \in \qq$, the set $\lbrace q \in \mbb P \mid v_q(x) \neq 0 \rbrace$ is finite.
\end{itemize}
\end{prop}
\begin{proof}
Exercise.
\end{proof}
One should think of $v_p$ as a map which measures how divisible an element is by $p$.
\begin{eg}
$2$ is not a square in $\qq$, here is a proof: suppose there would exist $x \in \qq$ with $x^2 = 2$.
Then $1 = v_2(2) = v_2(x^2) = 2v_2(x) \in 2\zz$.
Contradiction.
\end{eg}

\subsection{The $p$-adic topology}
We now present another way to think about the field of $p$-adic numbers $\qq_p$, which reveals an analogy with the field of real numbers $\rr$.

One way to show that a polynomial $f \in \qq[X_1, \ldots, X_n]$ does not have a zero in $\qq^n$, is by showing that it does not have a zero in $\rr^n$.
Since (by definition) real numbers can be approximated arbitrarily by rational numbers (in other words, $\qq$ is dense in $\rr$), and by the completeness of $\rr$, it follows that $f$ has a root in $\rr^n$ if and only if for every $m \in \nat$ there exists $x_m \in \qq^n$ such that $\lvert f(x_m) \rvert < 1/m$.

In fact, while there are many ways to construct the field $\rr$, it is completely characterised by the conditions that it is an ordered field, $\qq$ is dense in $\rr$, and $\rr$ is complete with respect to the metric induced by the absolute value, i.e.~every Cauchy sequence has a limit.

We now define a different metric on the field of rational numbers $\qq$ than the standard metric induced by the absolute value.
\begin{prop}\label{P:p-adic-metric}
Let $p \in \mbb P$.
Consider the map
$$ d_p : \qq_p \times \qq_p \to \rr_{\geq 0} : (x, y) \mapsto p^{-v_p(x-y)}$$
where we take the convention $p^{-\infty} = 0$.
Then $d_p$ defines a metric on $\qq_p$, i.e.~it satisfies for $x, y, z \in \qq_p$
\begin{enumerate}[(i)]
\item $d_p(x, y) = 0$ if and only if $x = y$,
\item $d_p(x, y) = d_p(y, x)$,
\item\label{it:strong-triangle} $d_p(x, z) \leq \max \lbrace d(x, y), d(y, z) \rbrace \leq d(x, y) + d(y, z)$.
\end{enumerate}
\end{prop}
\begin{proof}
This follows immediately from \Cref{P:p-adic-valuation}.
\end{proof}
\begin{defi}
The map $d_p$ defined in \Cref{P:p-adic-metric} is called the \emph{$p$-adic metric}\index{$p$-adic metric}.
The topology it induces on $\qq_p$ is called the \emph{$p$-adic topology}\index{$p$-adic topology}.
This map also induces a metric (and thus a topology) on $\zz_p$, $\qq$, and $\zz$, which we will also call the $p$-adic metric (respectively topology).
The first inequality in \eqref{it:strong-triangle} is called the \emph{strong triangle inequality}\index{strong triangle inequality}
\end{defi}
\begin{eg}
Consider the $p$-adic metric $d_p$ on $\qq_p$. The ball of radius $1$ around the origin is precisely $\zz_p$.
\end{eg}
As for any metric space, we can talk about convergence of sequences, Cauchy sequences, open and closed subsets, dense subsets, et cetera, for the $p$-adic metric.
The intuition should be that elements are close to eachother when their difference has high $p$-adic value, i.e.~is divisible in $\zz_p$ by a high power of $p$.
For example, one verifies that:
\begin{itemize}
\item given a sequence $(x_n)_n$ in $\qq_p$ and $x \in \qq_p$, we have that $(x_n)_n$ converges to $x$ (or $\lim_{n \to \infty} x_n = x$) if for all $m \in \nat$ there exists $n_0 \in \nat$ such that for all $n \geq n_0$ one has $v_p(x - x_n) > m$.
\item given a sequence $(x_n)_n$ in $\qq_p$, we have that $(x_n)_n$ is a Cauchy sequence if for all $m \in \nat$ there exists $n_0 \in \nat$ such that for all $n_1, n_2 \geq n_0$ one has $v_p(x_{n_1} - x_{n_2}) > m$.
\end{itemize}
\begin{prop}\label{P:field-topology}
The $p$-adic topology is a field topology, i.e.~$\lbrace x \rbrace$ is a closed set for any $x \in \qq_p$, and the maps
\begin{align*}
\qq_p \times \qq_p &\to \qq_p : (x, y) \mapsto x+y \\
\qq_p^{\times} \times \qq_p^{\times} &\to \qq_p^{\times} : (x, y) \mapsto x \cdot y
\end{align*}
are continuous.
\end{prop}
\begin{proof}
Exercise.
\end{proof}
We shall show that $\qq_p$ is, in fact, the completion of $\qq$ with respect to the $p$-adic topology.
%\begin{prop}
%Let $p \in \mbb P$.
%$\qq$ is dense in $\qq_p$ with respect to the $p$-adic topology.
%Furthermore, $\zz$ is dense in $\zz_p$.
%\end{prop}
%\begin{proof}
%
%\end{proof}
\begin{thm}\label{T:Qp-complete}
Let $p \in \mbb P$.
Then $\zz_p$ and $\qq_p$ are complete with respect to the $p$-adic metric, i.e. every Cauchy sequence in $\zz_p$ (respectively $\qq_p$) converges to an element of $\zz_p$ (respectively $\qq_p$).
\end{thm}
\begin{proof}
Let $(x_n)_n$ be a Cauchy sequence in $\zz_p$.
To show that $(x_n)_n$ converges, it suffices to show that there is a convergent subsequence.
By removing intermediate terms, we may thus assume without loss of generality that, for all $n\in\nat$ and for all $n_1, n_2 \geq n$, $v_p(x_{n_1} - x_{n_2}) \geq n$.
Write $x_n = (x_n^{(m)} + p^m\zz)_{m}$ for some $x_n^{(m)} \in \zz$.
Define $x = (x_m^{(m)} + p^m\zz)_{m}$.
We claim that $x \in \zz_p$ and $x = \lim_{n \to \infty} x_n$.

Consider $n \in \nat^+$.
For $m \geq n$ we have that $v_p(x_n - x_m) \geq n$, which implies that $x_n^{(m)} \equiv x_m^{(m)} \bmod p^n$.
Applying this to $m = n+1$ in particular we obtain $x_{n}^{(n+1)} \equiv x_{n}^{(n+1)} \equiv x_{n+1}^{(n+1)} \bmod p^n$; having this for general $n$ shows $x \in \zz_p$.
Furthermore, we see that $v_p(x_n - x) \geq n$ by construction.
This implies $\lim_{n \to \infty} x_n = x$ as desired.
We have shown the completeness of $\zz_p$.

The completeness of $\qq_p$ then follows from this: consider a Cauchy sequence $(x_n)_n$ in $\zz_p$.
There must exist some $k \in \nat$ such that $v_p(x_n) \geq -k$ for all $n \in \nat$.
But then $(p^kx_n)_n$ is a Cauchy sequence in $\zz_p$, which by the previous part converges to some $x' \in \zz_p$, but then $(x_n)_n$ converges to $p^{-k}x'$.
\end{proof}

We give another presentation of $p$-adic numbers using infinite series.
Just like with the real numbers, when $(x_n)_n$ is a sequence of $p$-adic numbers, we denote by $\sum_{n=0}^{+\infty} x_n$ the element $\lim_{n \to +\infty} \sum_{i=0}^n x_i$, \textit{assuming that this limit converges}.
\begin{prop}\label{P:series-expansion}
Let $p \in \mbb P$.
\begin{enumerate}[(i)]
\item\label{it:series-limit-exists} For any sequence $(a_n)_{n}$ in $\zz_p$, the sum $\sum_{n=0}^{+\infty} a_np^n$ converges in $\zz_p$.
\item\label{it:series-expansion-zzp} For any $x \in \zz_p$, there exists a unique sequence $(a_n)_n$ with $a_n \in \lbrace 0, 1, \ldots, p-1 \rbrace$ such that $x = \sum_{n=0}^{+\infty} a_np^n$.
\item\label{it:series-expansion-qqp} For any $x \in \qq_p$, there exists a unique sequence $(a_n)_{n=v_p(x)}^{+\infty}$ with $a_n \in \lbrace 0, 1, \ldots, p-1 \rbrace$ such that $x = \sum_{n = v_p(x)}^{+\infty}a_np^n$.
\end{enumerate}
\end{prop}
\begin{proof}
\eqref{it:series-limit-exists} follows immediately from the completeness of $\zz_p$ (\Cref{T:Qp-complete}).

\eqref{it:series-expansion-zzp}: Write $x = (x_n + p^n\zz)_n$ for $x_n \in \zz$.
Suppose that $x = \sum_{n=0}^{+\infty} a_np^n$ for some $a_n \in \lbrace 0, \ldots, p-1\rbrace$.
We shall determine what values the $a_n$ must necessarily have, which shall establish the uniqueness, and then simultaneously verify that one can always choose the $a_n$ as such, which establishes existence in view of \eqref{it:series-limit-exists}.

Since $x \equiv \sum_{n=0}^{+\infty} a_np^n \equiv a_0 \bmod p$, we must have $a_0 \equiv x_1 \bmod p$.
We thus choose $a_0$ as the unique element in $\lbrace 0, \ldots, p - 1 \rbrace$ with this property.
Now assume that $a_0, \ldots, a_{n-1}$ have been chosen such that $\sum_{i=0}^{n-1} a_ip^i \equiv x_n \bmod p^n$.
Since $x_{n+1} \equiv x_n \bmod p^n$, we have that $x_{n+1} = \sum_{i=0}^{n-1}a_ip^i + p^nb$ for some $b \in \zz$.
There is a unique $a_n \in \lbrace 0, \ldots, p-1\rbrace$ such that $a_n \equiv b \bmod p$, and then we have by construction $x_{n+1} \equiv \sum_{i=0}^n a_ip^i \bmod p^{n+1}$.

\eqref{it:series-expansion-qqp} follows from \eqref{it:series-expansion-zzp} and \Cref{P:Zp-properties}.
\end{proof}
For $x \in \zz_p$, its unique presentation as $\sum_{n=0}^{+\infty} a_np^n$ for $a_n \in \lbrace 0, 1, \ldots, p-1 \rbrace$ is called its \emph{$p$-adic series expansion}\index{$p$-adic series expansion} or simply \emph{$p$-adic expansion}.
\begin{eg}
We find the $7$-adic expansion of $\frac{142}{9}$.

Note that $\frac{142}{9} = 16 - \frac{2}{9}$.
We can easily find a $7$-adic expansion of $16$: we have $16 = 2\cdot 7^0 + 2 \cdot 7^1$.
To find a $7$-adic expansion of $\frac{1}{9}$, we shall use (see \eqref{ex:geometric-series} below) that, for any $k \in \nat^+$, we have in $\zz_7$ that
$$ \frac{1}{1-7^k} = \sum_{i=0}^{+\infty} 7^{ik}.$$
Since $9$ and $7$ are coprime, there must exist $k \in \nat^+$ such that $9 \mid (7^k - 1)$; one verifies that $k = 3$ works.
We compute that
$$\frac{1}{9} = \frac{1}{9} \cdot \frac{1 - 7^3}{1 - 7^3} = \frac{-38}{1 - 7^3} = -38\sum_{n=0}^{+\infty} 7^{3n}$$
and thus
\begin{align*}
\frac{142}{9} &= 16 - \frac{2}{9} = 2 \cdot 7^0 + 2 \cdot 7^1 + 76 \sum_{n=0}^{+\infty} 7^{3n} \\
&= 2\cdot 7^0 + 2 \cdot 7^1 + (6\cdot 7^0 + 3 \cdot 7^1 + 1 \cdot 7^2)\sum_{n=0}^{+\infty} 7^{3n} \\
&= 2 \cdot 7^0 + 2 \cdot 7^1 + 6\cdot 7^0 + 6\sum_{n=1}^{+\infty} 7^{3n} + 3\cdot 7^1 + 3\sum_{n=1}^{+\infty} 7^{3n+1} + \sum_{n=0}^{+\infty}7^{3n+2} \\
&= 1 \cdot 7^0 + 6 \cdot 7^1 + \sum_{n=0}^{+\infty} (7^{3n+2} + 6 \cdot 7^{3n+3} + 3 \cdot 7^{3n+4}).
\end{align*}
We see that the $7$-adic representation of $\frac{142}{9}$ becomes periodic after finitely many terms.
This is no coincide, see Exercise \eqref{ex:p-adic-periodic} below.
\end{eg}
\begin{cor}
$\zz$ is dense in $\zz_p$ and $\qq$ is dense in $\qq_p$ with respect to the $p$-adic topology.
\end{cor}
\begin{proof}
This is immediate from parts \eqref{it:series-expansion-zzp} and \eqref{it:series-expansion-qqp} of \Cref{P:series-expansion}.
\end{proof}
Just like for the reals, one can show that $\qq_p$ is uniquely determined up to canonical isomorphism by the property that it is a field with a complete metric extending the $p$-adic metric on $\qq$ and in which $\qq$ is dense; see e.g.~\cite[Theorem 2.4.3]{Eng05}.

\subsection{Exercises}
Let always $p \in \mbb P$.
\begin{enumerate}
\item Prove \Cref{P:Zp-construction}, \Cref{P:p-adic-valuation}, and \Cref{P:field-topology} and fill in the missing details in the proof of \Cref{P:Zp-properties}.
\item 
Consider the construction of $\zz_p$ given in \Cref{P:Zp-construction}, but instead of considering $p \in \mbb P$, we do the same construction for $p = 10$.
Show, by giving explicit zero divisors, that $\zz_{10}$ is not an integral domain.
\item\label{ex:formal-Taylor} Let $R$ be a commutative ring, $f \in R[X]$ a polynomial of degree at most $n$, $a \in R$.
Show that
\begin{displaymath}
f(X) = f(a) + \sum_{i=1}^n \frac{f^{(i)}(a)(X-a)^i}{i!}
\end{displaymath}
where $f^{(i)}$ denotes the $i$th formal derivative of $f$ (i.e.~$f^{(i+1)} = (f^{(i)})'$).
\item Consider the polynomial $f(X) = 10X^4 + 3X^3 - 3X^2 + 6$.
Determine whether $f$ has a root in $\zz_2$, and whether $f$ has a root in $\zz_3$.
\item\label{ex:geometric-series} Show that, for $k \in \nat^+$, one has $\sum_{i=0}^{+\infty} p^{ki} = (1-p^k)^{-1}$ in $\zz_p$. (\textit{Hint:} use the fact that $\lim_{n \to \infty} p^n = 0$ in $\zz_p$.)
\item\label{ex:p-adic-periodic} Let $x \in \qq_p$.
Show that $x \in \qq$ if and only if the $p$-adic expansion of $x$ is eventually periodic, i.e.~writing $x = \sum_{n=k}^{+\infty} a_np^n$ for some $k \in \zz$, $a_n \in \lbrace 0, 1, \ldots, p-1 \rbrace$, there exists $n_0, m \in \nat^+$ such that for all $n \geq n_0$, $a_{n+m} = a_n$.
\item Given $x = \sum_{n=0}^{+\infty} a_np^n$ for certain $a_n \in \lbrace 0, \ldots, p-1\rbrace$.
Find a formula for the $p$-adic expansion of $-x$.
\item Show that the $p$-adic topology on $\zz_p$ is compact.
\item\label{ex:solution-patching} Let $n \in \nat$ and $f \in \zz_p[X_1, \ldots, X_n]$.
Assume that, for all $m \in \nat$, there exists $(x_1, \ldots, x_n) \in \zz^n$ with $f(x_1, \ldots, x_n) \equiv 0 \bmod p^m$.
Use the Axiom of Choice to infer that there exists $(x_1, \ldots, x_n) \in \zz_p^n$ with $f(x_1, \ldots, x_n) = 0$.
\item Show that $\zz_p$ is uncountable.
\end{enumerate}

\end{document}