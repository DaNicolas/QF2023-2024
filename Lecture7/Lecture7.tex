\documentclass[12pt, leqno, british]{amsart}
\usepackage[style=alphabetic, backend=biber]{biblatex}
\usepackage{a4, amsmath}
\usepackage{mathtools}
\usepackage{amssymb}
\usepackage{amsthm, amscd, mathdots}
\swapnumbers
\usepackage{enumerate}
\usepackage{hyperref}
\usepackage{cleveref}
\usepackage{csquotes}
\usepackage{color}
\usepackage{datetime}
\usepackage{xr, standalone, import}

\theoremstyle{definition}
\newtheorem{defi}{Definition}[subsection]
\theoremstyle{plain}
\newtheorem{prop}[defi]{Proposition}
\newtheorem{lem}[defi]{Lemma}
\newtheorem{thm}[defi]{Theorem}
\newtheorem{cor}[defi]{Corollary}
\newtheorem{ques}[defi]{Question}
\theoremstyle{remark}
\newtheorem{rem}[defi]{Remark}
\newtheorem{eg}[defi]{Example}
\newtheorem{egs}[defi]{Examples}

\newcommand{\mc}{\mathcal}
\newcommand{\mf}{\mathfrak}
\newcommand{\mbb}{\mathbb}
\newcommand{\nat}{\mbb N}
\newcommand{\cc}{\mathbb C}
\newcommand{\rr}{\mathbb R}
\newcommand{\qq}{\mbb Q}
\newcommand{\ovl}{\overline}
\newcommand{\ff}{\mbb F}
\newcommand{\zz}{\mbb Z}

\DeclareMathOperator{\charac}{char}
\DeclareMathOperator{\id}{id}
\DeclareMathOperator{\Frac}{Frac}
\DeclareMathOperator{\Ker}{Ker}
\DeclareMathOperator{\Img}{Im}
\DeclareMathOperator{\Trd}{Trd}
\DeclareMathOperator{\Tr}{Tr}
\DeclareMathOperator{\Nrd}{Nrd}
\DeclareMathOperator{\GL}{GL}
\DeclareMathOperator{\Gal}{Gal}
\DeclareMathOperator{\ord}{ord}
\DeclareMathOperator{\trdeg}{trdeg}
\DeclareMathOperator{\supp}{supp}
\DeclareMathOperator{\rad}{rad}
\DeclareMathOperator{\sign}{sign}
\newcommand{\disc}{\mathrm{d}}

\newcommand{\llangle}{\langle\!\langle}
\newcommand{\rrangle}{\rangle\!\rangle}
\addbibresource{../bibliography.bib}
\externaldocument[M-]{../Lecture-notes}

\author{Nicolas Daans}
\address{Charles University, Faculty of Mathematics and Physics, Department of Algebra, Sokolov\-sk\' a 83, 18600 Praha~8, Czech Republic.}
\email{nicolas.daans@matfyz.cuni.cz}

\begin{document}

\section{Lecture 7}

\subsection{Squares in $p$-adic fields}
The goal of this lecture and the next will be to completely classify quadratic forms over $\qq_p$ for each $p \in \mbb{P}$.
We start by studying when an element of $\qq_p$ is a square.
In view of \Cref{M-P:disc-map}, this will allow us to compute $I\qq_p/I^2\qq_p \cong \qq_p^{\times}/\qq_p^{\times 2}$.

\begin{prop}[Local Square Theorem]\label{LocalSquareThm}
Let $p\in\mathbb{P}$ and $\alpha\in\mathbb{Z}_{p}$. Then $1+4p\alpha\in\mathbb{Z}_{p}^{2}$, that is, $1+4p\alpha$ is a square of a number from $\mathbb{Z}_{p}$.

In particular, if $p>2$ and $\lambda\in p\mathbb{Z}_{p}$, then $1+\lambda\in\mathbb{Z}_{p}^{2}$.
\end{prop}
\begin{proof}
Consider $f(x)=px^{2} + x - \alpha\in\mathbb{Z}_{p}[x]$. Then
\begin{align*}
f(\alpha ) = p\alpha^{2} \equiv 0 \pmod{p} \hspace{0.3cm} \textrm{ and } \hspace{0.3cm} f'(\alpha ) =2p\alpha + 1\equiv 1\not\equiv 0 \pmod{p},
\end{align*}
so we can apply Hensel's Lemma (Corollary \ref{Hensel'sLemma2}). We get that there exists $\beta\in\mathbb{Z}_{p}$ such that $f(\beta )=0$. By the quadratic formula, 
\begin{align*}
\beta = \frac{-1 \pm \sqrt{1+4p\alpha}}{2p},
\end{align*}
so $1+4p\alpha = (1+2p\beta )^{2} \in\mathbb{Z}_{p}^{2}$ as desired.

For the ``In particular'' part, it is enough to apply the main part for $\alpha =\frac{\lambda}{4p}$.
\end{proof}
\begin{cor}\label{C:qqp-squares-odd}
Let $p \in \mbb P \setminus \lbrace 2 \rbrace$.
Let $u \in \mbb Z_p$ such that $\ovl u$ is not a square in $\ff_p$.
We have that $\lvert \qq_p^\times / \qq_p^{\times 2}\rvert = 4$; more specifically, $\qq_p^\times / \qq_p^{\times 2} = \lbrace \qq_p^{\times 2}, u\qq_p^{\times 2}, p\qq_p^{\times 2}, up\qq_p^{\times 2} \rbrace$.
\end{cor}
\begin{proof}
We first show that $u$, $p$, and $up$ are not squares in $\qq_p$.
If we had $p = \alpha^2$ for some $\alpha \in \qq_p$, then $1 = v_p(p) = v_p(\alpha^2) = 2v_p(\alpha)$, which is not possible, since $v_p(\alpha)$ must be an integer.
Thus, $p$ is not a square in $\qq_p$, and by a similar argument, $up$ is not a square in $\qq_p$.
Assume now that $u = \alpha^2$ for some $\alpha \in \qq_p$.
Then $0 = v_p(u) = 2v_p(\alpha)$, so, $v_p(\alpha) = 0$ and hence $\alpha \in \zz_p$.
The equation $u = \alpha^2$ then implies $\ovl{u} = \ovl{\alpha}^2$ in $\ff_p$, but this is impossible, since we assumed that $\ovl{u}$ is not a square in $\ff_p$.

Now, we need to show that an arbitrary non-zerp element of $\qq_p$ can be written as a square times $1$, $u$, $p$, or $up$.
So take $\alpha \in \qq_p^\times$ arbitrary.
Write $\alpha = p^k \alpha'$ for some $k \in \zz$ and $\alpha' \in \zz_p^\times$.
Since $\lvert \ff_p^\times / \ff_p^{\times 2} \rvert = 2$ (see exercises Lecture 4) and $\ovl{u} \not\in \ff_p^{\times 2}$, we have that either $\ovl{\alpha'} \in \ff_p^{\times 2}$ or $\ovl{\alpha'u} \in \ff_p^{\times 2}$.
Set $\alpha'' = \alpha'$ in the first case, or $\alpha'' = \alpha'u$ in the second case.

We now show that $\alpha'' \in \qq_p^{\times 2}$.
Since $\alpha$ can be written as $\alpha''$ multiplied with a product of powers of $p$ and $u$, this will conclude the proof that every element of $\qq^\times$ is a square times $1$, $p$, $u$, or $up$.
Since $\ovl{\alpha''} \in \ff_p^{\times 2}$, we can find $\beta \in \zz_p^\times$ such that $\ovl{\alpha'} = \ovl{\beta}^2$.
Then $\ovl{\alpha''(\beta^{-1})^2} = 1$, whereby $\alpha''(\beta^{-1})^2 \in 1 + p\zz_p$.
By \Cref{LocalSquareThm} we conclude that $\alpha''(\beta^{-1})^2 \in \qq_p^{\times 2}$ and hence $\alpha'' \in \qq_p^{\times 2}$, as desired.
\end{proof}
\begin{cor}\label{C:qq2-squares}
We have that $\lvert \qq_2^\times/\qq_2^{\times 2} \rvert = 8$; more specifically, $\qq_2^\times/\qq_2^{\times 2} = \lbrace \pm\qq_2^{\times 2}, \pm 2\qq_2^{\times 2}, \pm 3\qq_2^{\times 2}, \pm 6 \qq_2^{\times 2} \rbrace$.
\end{cor}
\begin{proof}
We proceed as in the proof of \Cref{C:qqp-squares-odd}.
First, we show that $-1, 2, -2, 3, -3, 6, -6$ are all non-squares in $\qq_2$.
For $\alpha = 2, -2, 6, -6$ we have $v_2(\alpha)=1$, so we conclude as before that $\alpha$ cannot be a square.
For $\alpha = -1, 3, -3$, assume that $\alpha = \beta^2$ for some $\beta \in \qq_2$.
Then $0 = v_2(\alpha) = 2v_2(\beta)$, so $v_2(\beta) = 0$ and thus $\beta \in \zz_2$.
Since $\alpha = \beta^2$, we have in particular $\alpha \equiv \beta^2 \bmod 8\zz_2$.
But it is easy to compute that in $\zz_2/8\zz_2 \cong \zz/8\zz$, $-1$, $3$ and $-3$ are not squares.

Now, we need to show that an arbitrary non-zero element of $\qq_2$ can be written as a square times $\pm 1$, $\pm 2$, $\pm 3$, or $\pm 6$.
This is similar as in the proof of \Cref{C:qqp-squares-odd} and left as an exercise.
\end{proof}

\subsection{$2$-fold Pfister forms over $p$-adic fields}
Now that we know what the group $I\qq_p/I^2\qq_p \cong \qq_p^\times/\qq_p^{\times 2}$ looks like, we look at the next quotient $I^2\qq_p/I^3\qq_p$.
In view of \Cref{M-P:symb-equal}, this means we have to study $2$-fold Pfister forms over $\qq_p$.
We shall see that $I^2\qq_p/I^3\qq_p \cong \zz/2\zz$; in other words, there is a unique anisotropic $2$-fold Pfister form over $\qq_p$ up to isometry.
We shall compute this Pfister form explicitly, and more generally, see how to decide when a $2$-fold Pfister form over $\qq_p$ is isotropic.

\begin{lem}\label{L:isotropic-vector-in-zzp}
Let $q : \qq_p^n \to \qq_p$ be an isotropic quadratic form over $\qq_p$.
There exists $v = (x_1, \ldots, x_n) \in \zz_p^n$ such that $q(v) = 0$ and there exists $i \in \lbrace 1, \ldots, n \rbrace$ with $x_i \in \zz_p^\times$.
\end{lem}
\begin{proof}
As $q$ is isotropic, there exists $0 \neq w = (y_1, \ldots, y_n) \in \qq_p^n$ with $q(w) = 0$.
Let $k = \min \lbrace v_p(y_1), \ldots, v_p(y_n) \rbrace$.
Setting $x_i = p^{-k}y_i$ and $v = (x_1, \ldots, x_n)$, we compute that $q(v) = p^{-2k}q(w) = 0$ and $\min \lbrace v_p(x_1), \ldots, v_p(x_n) \rbrace = 0$, so this vector is as desired.
\end{proof}
As will often be the case, the case $p\neq 2$ is the easiest, so we start with that.
\begin{lem}
Let $p \in \mbb P \setminus \lbrace 2 \rbrace$ and consider $a_1, a_2, a_3 \in \zz_p^\times$.
Then
\begin{itemize}
\item $\langle a_1, a_2 \rangle_{\qq_p}$ is isotropic if and only if $\ovl{-a_1a_2}$ is a square in $\ff_p$.
If it is anisotropic, then for any $x_1, x_2 \in \qq_p$ we have
\begin{displaymath}
v(a_1x_1^2 + a_2x_2^2) = 2\min \lbrace v(x_1), v(x_2) \rbrace.
\end{displaymath}
\item $\langle a_1, a_2, a_3 \rangle_{\qq_p}$ is always isotropic.
\end{itemize}
\end{lem}
\begin{proof}

\end{proof}
\begin{lem}\label{L:qqp-odd-4d-form}
Let $p \in \mbb P \setminus \lbrace 2 \rbrace$ and $a_1, a_2, a_3, a_4 \in \zz_p^\times$.
Then
\begin{itemize}
\item $\langle a_1, a_2, pa_3 \rangle_{\qq_p}$ is anisotropic if and only if $\overline{-a_1a_2}$ is a non-square in $\ff_p$.
\item $\langle a_1, a_2, pa_3, pa_4 \rangle_{\qq_p}$ is anisotropic if and only if $\overline{-a_1a_2}$ and $\overline{-a_3a_4}$ are both non-squares in $\ff_p$.
\end{itemize}
\end{lem}
\begin{proof}

\end{proof}
\begin{prop}\label{P:2-fold-Pfister-qqp-odd}
Let $p \in \mbb P \setminus \lbrace 2 \rbrace$ and $a_1, a_2 \in \qq_p^ \times$.
Write $a_1 = p^{k_1}u_1$ and $a_2 = p^{k_2}u_2$ for $k_1, k_2 \in \zz$ and $u_1, u_2 \in \zz_p^\times$.
The $2$-fold Pfister form $\llangle a_1, a_2 \rrangle_{\qq_p}$ is anisotropic if and only if one of the following occurs:
\begin{itemize}
\item $k_1$ is even, $k_2$ is odd, and $\ovl{u_1} \not\in \ff_p^{\times 2}$,
\item $k_1$ is odd, $k_2$ is even, and $\ovl{u_2} \not\in \ff_p^{\times 2}$,
\item $k_1$ is odd, $k_2$ is odd, and $\ovl{-u_1u_2} \not\in \ff_p^{\times 2}$.
\end{itemize}
\end{prop}
\begin{proof}

\end{proof}
\begin{cor}\label{C:unique-2-fold-Pfister-qqp-odd}
Let $p \in \mbb P \setminus \lbrace 2 \rbrace$, let $u \in \zz_p^\times$ such that $\ovl{u} \not\in \ff_p^{\times 2}$.
Then $\llangle u, p \rrangle_{\qq_p}$ is the unique anisotropic $2$-fold Pfister form over $\qq_p$ up to isometry.
In particular, $I\qq_p/I^2\qq_p \cong \zz/2\zz$.
\end{cor}
\begin{proof}
By \Cref{P:2-fold-Pfister-qqp-odd} the quadratic form $\llangle u, p \rrangle_{\qq_p}$ is anisotropic.

It remains to show that every element of $I\qq_p/I^2\qq_p = \lbrace 0, \lbrace u, p \rbrace_{\qq_p} \rbrace$.
By \Cref{C:qqp-squares-odd} and in view of \Cref{M-C:K-theory-axioms}\eqref{M-it:scale-square-invariance} it suffices to show that $\lbrace a_1, a_2 \rbrace_{\qq_p} \in \lbrace 0, \lbrace u, p \rbrace_{\qq_p} \rbrace$ where $a_1, a_2 \in \lbrace 1, u, p, up \rbrace$.
This is now easily computed as follows, using the computation rules from \Cref{M-P:K-theory-axioms} and \Cref{M-C:K-theory-axioms}.
Note that either $-1 \in \qq_p^{\times 2}$ or $-1 \in u\qq_p^{\times 2}$ (depending on whether $\overline{-1} \in \ff_p^{\times 2}$, so $\lbrace -1, p \rbrace_{\qq_p}$ is either equal to $0$ or to $\lbrace u, p \rbrace_{\qq_p}$).
\begin{itemize}
\item If $a_1 = 1$, or if $a_2 = 1$, or if $a_1 = a_2 = u$, then $\lbrace a_1, a_2 \rbrace_{\qq_p} = 0$ by \Cref{L:qqp-odd-4d-form},
\item $\lbrace p, u \rbrace_{\qq_p} = \lbrace u, p \rbrace_{\qq_p}$,
\item $\lbrace up, u \rbrace_{\qq_p} = \lbrace u, up \rbrace_{\qq_p} = \lbrace u, u \rbrace_{\qq_p} + \lbrace u, p \rbrace_{\qq_p} = 0 + \lbrace u, p \rbrace_{\qq_p} = \lbrace u, p \rbrace_{\qq_p}$,
\item $\lbrace p, p \rbrace_{\qq_p} = \lbrace 2p, -p^2 \rbrace_{\qq_p} = \lbrace 2p, -1 \rbrace_{\qq_p} = \lbrace 2, -1 \rbrace_{\qq_p} + \lbrace p, -1 \rbrace_{\qq_p} = 0 + \lbrace -1, p \rbrace_{\qq_p} \in \lbrace 0, \lbrace u, p \rbrace_{\qq_p} \rbrace$,
\item $\lbrace p, up \rbrace_{\qq_p} = \lbrace up, p \rbrace_{\qq_p} = \lbrace u, p \rbrace_{\qq_p} + \lbrace p, p \rbrace_{\qq_p} \in \lbrace 0, \lbrace u, p \rbrace_{\qq_p} \rbrace,$
\item $\lbrace up, up \rbrace_{\qq_p} = \lbrace 2up, -(up)^2 \rbrace_{\qq_p} = \lbrace 2, -1 \rbrace_{\qq_p} + \lbrace up, -1 \rbrace_{\qq_p} \in \lbrace \lbrace up, 1 \rbrace_{\qq_p}, \lbrace up, u \rbrace_{\qq_p} \rbrace.$
\end{itemize}
This concludes the proof.
\end{proof}
We now discuss the case $p=2$.
\begin{prop}\label{P:2-fold-Pfister-qq2}
$\llangle -1, -1 \rrangle_{\qq_2}$ is the unique anisotropic $2$-fold Pfister form over $\qq_2$ up to isometry.
In particular, $I\qq_2/I^2\qq_2 \cong \zz/2\zz$.
Furthermore, the following table shows for which values of $a_1, a_2 \in \lbrace \pm 1, \pm 2, \pm 3, \pm 6 \rbrace$ the form $\llangle a_1, a_2 \rrangle_{\qq_p}$ is isotropic ($0$) or anisotropic ($1$).

\centering
\begin{tabular}{r | c c c c c c c c}
$a/b$ & 1 & -1 & 2 & -2 & 3 & -3 & 6 & -6 \\ \hline
1 & 0 & 0 & 0 & 0 & 0 & 0 & 0 & 0 \\
-1 & 0 & 1 & 0 & 1 & 1& 0 & 1 & 0 \\
2 & 0 & 0 & 0 & 0 & 1 & 1 & 1 & 1 \\
-2 & 0 & 1 & 0 & 1 & 0 & 1 & 0 & 1 \\
3 & 0 & 1 & 1 & 0 & 1 & 0 & 0 & 1\\
-3 & 0 & 0 & 1 & 1 & 0 & 0 & 1 & 1 \\
6 & 0 & 1 & 1 & 0 & 0 & 1 & 1 & 0 \\
-6 & 0 & 0 & 1 & 1 & 1 & 1 & 0 & 0
\end{tabular}
\end{prop}
\begin{proof}
To see that $\llangle -1, -1 \rrangle_{\qq_2}$ is anisotropic, suppose for the sake of a contradiction that it is isotropic.
By \Cref{L:isotropic-vector-in-zzp} this means there exist $x_1, \ldots, x_4 \in \zz_p$ with $x_1^2 + \ldots + x_4^2 = 0$ and (without loss of generality) $x_1 \in \zz_p^\times$.
But then we have $\ovl{x_1}^2 + \ldots + \ovl{x_4}^2 = 0$ in $\zz_2/8\zz_2 \cong \zz/8\zz$ and $\ovl{x_1} \neq 0$; one verifies that this is impossible.

To verify the table, and to prove that $I\qq_2/I^2\qq_2 = \lbrace 0, \lbrace -1, -1 \rbrace_{\qq_2} \rbrace$, we compute $\lbrace a_1, a_2 \rbrace$ for $a_1, a_2 \in \lbrace \pm 1, \pm 2, \pm 3, \pm 6 \rbrace$. This is sufficient, given \Cref{C:qqp-squares-odd}.
But this table can be established entirely via the computation rules of \Cref{M-P:K-theory-axioms} and \Cref{M-P:K-theory-axioms}.
\end{proof}

\subsection{Exercises}
\begin{enumerate}
\item Complete the proof of \Cref{C:qq2-squares}.
\item Complete the proof of \Cref{P:2-fold-Pfister-qq2}.
\item Determine completely the set of all prime numbers $p$ for which the quadratic form $\llangle 15, 33 \rrangle_{\qq_p}$ is anisotropic.
\end{enumerate}

\end{document}