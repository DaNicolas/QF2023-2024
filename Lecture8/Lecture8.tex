\documentclass[12pt, leqno, british]{amsart}
\usepackage[style=alphabetic, backend=biber]{biblatex}
\usepackage{a4, amsmath}
\usepackage{mathtools}
\usepackage{amssymb}
\usepackage{amsthm, amscd, mathdots}
\swapnumbers
\usepackage{enumerate}
\usepackage{hyperref}
\usepackage{cleveref}
\usepackage{csquotes}
\usepackage{color}
\usepackage{datetime}
\usepackage{xr, standalone, import}

\theoremstyle{definition}
\newtheorem{defi}{Definition}[subsection]
\theoremstyle{plain}
\newtheorem{prop}[defi]{Proposition}
\newtheorem{lem}[defi]{Lemma}
\newtheorem{thm}[defi]{Theorem}
\newtheorem{cor}[defi]{Corollary}
\newtheorem{ques}[defi]{Question}
\theoremstyle{remark}
\newtheorem{rem}[defi]{Remark}
\newtheorem{eg}[defi]{Example}
\newtheorem{egs}[defi]{Examples}

\newcommand{\mc}{\mathcal}
\newcommand{\mf}{\mathfrak}
\newcommand{\mbb}{\mathbb}
\newcommand{\nat}{\mbb N}
\newcommand{\cc}{\mathbb C}
\newcommand{\rr}{\mathbb R}
\newcommand{\qq}{\mbb Q}
\newcommand{\ovl}{\overline}
\newcommand{\ff}{\mbb F}
\newcommand{\zz}{\mbb Z}

\DeclareMathOperator{\charac}{char}
\DeclareMathOperator{\id}{id}
\DeclareMathOperator{\Frac}{Frac}
\DeclareMathOperator{\Ker}{Ker}
\DeclareMathOperator{\Img}{Im}
\DeclareMathOperator{\Trd}{Trd}
\DeclareMathOperator{\Tr}{Tr}
\DeclareMathOperator{\Nrd}{Nrd}
\DeclareMathOperator{\GL}{GL}
\DeclareMathOperator{\Gal}{Gal}
\DeclareMathOperator{\ord}{ord}
\DeclareMathOperator{\trdeg}{trdeg}
\DeclareMathOperator{\supp}{supp}
\DeclareMathOperator{\rad}{rad}
\DeclareMathOperator{\sign}{sign}
\newcommand{\disc}{\mathrm{d}}

\newcommand{\llangle}{\langle\!\langle}
\newcommand{\rrangle}{\rangle\!\rangle}
\addbibresource{../bibliography.bib}
\externaldocument[M-]{../Lecture-notes}

\author{Nicolas Daans}
\address{Charles University, Faculty of Mathematics and Physics, Department of Algebra, Sokolov\-sk\' a 83, 18600 Praha~8, Czech Republic.}
\email{nicolas.daans@matfyz.cuni.cz}

\begin{document}

\section{Lecture 8}

\subsection{Classification of quadratic forms over $p$-adic fields}
We are ready to completely classify the quadratic forms over $p$-adic fields up to isometry.
Recall that, by Witt Decomposition (\Cref{M-T:Witt-Decomposition}), it suffices to classify the anisotropic quadratic forms up to isometry.

The $1$-dimensional case is immediate: over any field $K$, the quadratic forms $\langle a \rangle_K$ and $\langle b \rangle_K$ are isometric (for $a, b \in K^\times$) if and only if $ab \in K^{\times 2}$.
The classification of $1$-dimensional quadratic forms over $\qq_p$ thus follows from \Cref{M-C:qqp-squares-odd} and \Cref{M-C:qq2-squares}.

\begin{prop}\label{P:classify-2-dimensional-qqp}
Let $p \in \mbb P$.
Let $a_1, a_2, b_1, b_2 \in \qq_p^\times$.
Then $\langle a_1, a_2 \rangle_{\qq_p} \cong \langle b_1, b_2 \rangle_{\qq_p}$ if and only if $a_1a_2b_1b_2 \in \qq_p^{\times 2}$ and $\lbrace a_1a_2, a_1b_1 \rbrace_{\qq_2} = 0$.
\end{prop}
\begin{proof}
This follows from Exercise \eqref{M-ex:2-dim-form-isometric} of Lecture 4.
\end{proof}

\begin{prop}\label{P:classify-3-dimensional-qqp}
Let $p \in \mbb P$.
Let $a_1, a_2, a_3, b_1, b_2, b_3 \in \qq_p^\times$.
Then $\langle a_1, a_2, a_3 \rangle_{\qq_p}$ is isotropic if and only if $\lbrace -a_1a_2, -a_1a_3 \rbrace_{\qq_p} = 0$.
Furthermore, if $\langle a_1, a_2, a_3 \rangle_{\qq_p}$ and $\langle b_1, b_2, b_3 \rangle_{\qq_p}$ are both anisotropic, then we have $b_1b_2b_3\langle a_1, a_2, a_3 \rangle_{\qq_p} \cong a_1a_2a_3\langle b_1, b_2, b_3 \rangle_{\qq_p}$; in particular we have $\langle a_1, a_2, a_3 \rangle_{\qq_p} \cong \langle b_1, b_2, b_3 \rangle_{\qq_{p}}$ if and only if $a_1a_2a_3b_1b_2b_3 \in \qq_p^{\times 2}$.
\end{prop}
\begin{proof}
Observe that $\langle a_1, a_2, a_3 \rangle_{\qq_p} \cong a_1a_2a_3\langle a_1a_2, a_1a_3, a_2a_3\rangle_{\qq_p}$.

We have that $\lbrace -a_1a_2, -a_1a_3 \rbrace_{\qq_p} = 0$ if and only if $\llangle -a_1a_2, -a_1a_3 \rrangle_{\qq_p}$ is isotropic (\Cref{M-P:symb-equal}), if and only if $\llangle -a_1a_2, -a_1a_3 \rrangle_{\qq_p}$ is hyperbolic (\Cref{M-T:Pfister-forms}), if and only if its $3$-dimensional subform $\langle a_1a_2, a_1a_3, a_2a_3 \rangle_{\qq_p}$ is isotropic, if and only if $\langle a_1, a_2, a_3 \rangle_{\qq_p}$ is isotropic.
This shows the first statement.

Now assume that $\langle a_1, a_2, a_3 \rangle_{\qq_p}$ and $\langle b_1, b_2, b_3 \rangle_{\qq_p}$ are both anisotropic; this implies by the previous paragraph that $\lbrace -a_1a_2, -a_1a_3 \rbrace_{\qq_p}$ and $\lbrace -b_1b_2, -b_1b_3 \rbrace_{\qq_p}$ are both non-zero, but then by \Cref{M-C:unique-2-fold-Pfister-qqp-odd} or \Cref{M-P:2-fold-Pfister-qq2} they must be equal.
This implies in turn via \Cref{M-P:symb-equal} that $\llangle -a_1a_2, -a_1a_3 \rrangle_{\qq_p} \cong \llangle -b_1b_2, -b_1b_3 \rrangle_{\qq_p}$.
By Witt Cancellation (\Cref{M-T:Witt-Cancellation}) we conclude that $\langle a_1a_2, a_1a_3, a_2a_3\rangle_{\qq_p} \cong \langle b_1b_2, b_1b_3, a_2a_3\rangle_{\qq_p}$.
The rest of the statement follows by comparing determinants.
\end{proof}

\begin{prop}\label{P:classify-4-dimensional-qqp}
Let $p \in \mbb P$.
There is a unique anisotropic $4$-dimensional quadratic form over $\qq_p$ up to isometry.
This form is universal.
\end{prop}
\begin{proof}
We know from \Cref{M-C:unique-2-fold-Pfister-qqp-odd} or \Cref{M-P:2-fold-Pfister-qq2} that there exists a unique anisotropic $2$-fold Pfister form over $\qq_p$.
Let us write this as $\llangle a_1, a_2 \rrangle_{\qq_p}$, and recall that we may choose $a_1 = a_2 = -1$ if $p = 2$, and if $p \neq 2$, then choose $a_2 = p$ and $a_1 = u \in \zz_p^{\times}$ such that $\overline{u} \not\in \ff_p^{\times 2}$.

%We know from \Cref{P:classify-3-dimensional-qqp} that every anisotropic $3$-dimensional quadratic form over $\qq_p$ is of the form $d\langle -a_1, -a_2, a_1a_2 \rangle_{\qq_p}$ for some $d \in \qq_p^{\times}$.
One verifies that $\langle -a_1, -a_2, a_1a_2 \rangle_{\qq_p}$ represents all square classes in $\qq_p^\times/\qq_p^{\times 2}$ except for $-\qq^{\times 2}$: if $p = 2$ one checks by hand that each of $1, \pm 2, \pm 3, \pm 6$ are represented by $\langle 1, 1, 1 \rangle_{\qq_2}$, and if $p > 2$, one also verifies that $\langle -u, -p, up \rangle_{\qq_p}$ represents $u$, $p$ and $up$ if $-1 \in \qq_p^{\times 2}$, or that it represents $1$, $p$ and $up$ if $-1 \in u\qq_p^{\times 2}$.
%It follows that every $3$-dimensional anisotropic quadratic form over $\qq_p$ represents all square classes of $\qq_p$ except for minus its determinant.

Now, take an arbitrary anisotropic $4$-dimensional quadratic form $q$ over $\qq_p$.
As $\langle a_1, a_2 \rrangle_{\qq_p}$ is the unique anisotropic $2$-fold Pfister form over $\qq_p$, we have that $q \cong c\langle d, -a_1, -a_2, a_1a_2 \rangle_K$ for some $c, d \in K^\times$.
Furthermore, $-d$ is not represented by $\langle -a_1, -a_2, a_1a_2 \rangle_{\qq_p}$, so by the previous paragraph, we may assume $d = 1$.
Hence $q \cong c\llangle a_1, a_2 \rrangle_{\qq_p}$.
Furthermore, one sees similarly as in the previous paragraph that $\llangle a_1, a_2 \rrangle_{\qq_p}$ is universal, in particular it represents $c$, and hence $q \cong \llangle a_1, a_2 \rrangle_{\qq_p}$ since Pfister forms are multiplicative (\Cref{{M-T:Pfister-forms}}).
We have shown that every $4$-dimensional quadratic form over $\qq_p$ is isometric to $\llangle a_1, a_2 \rrangle_{\qq_p}$, and that this form is universal.
\end{proof}

\begin{cor}\label{C:classify-5-dimensional-qqp}
Let $p \in \mbb P$.
Every $5$-dimensional quadratic form over $\qq_p$ is isotropic.
\end{cor}
\begin{proof}
Let $q$ be a $5$-dimensional quadratic form over $\qq_p$.
We have $q \cong \langle a \rangle_{\qq_p} \perp q'$ for some $a \in \qq_p^\times$ and a $4$-dimensional quadratic form $q'$ by \Cref{M-P:diagonalisation}.
If $q$ would be anisotropic, then also $q'$ would be anisotropic, and then by \Cref{P:classify-4-dimensional-qqp} it universal; in particular it represents $-a$.
But then $q$ must actually have been isotropic.
\end{proof}

\subsection{Quadratic forms under field extensions}
For a field extension $L/K$ and a $K$-vector space $V$, the vector space $V_L = V \otimes L$ naturally becomes an $L$-vector space, with $\dim_K(V) = \dim_L(V_L)$ - see \Cref{M-P:tensor-product-properties}.
Furthermore, via the embedding $V \to V \otimes L : v \mapsto v \otimes 1$, we may identify $V$ with a $K$-subspace of $V \otimes L$.
We will now see that this gives a natural way to `extend' symmetric bilinear and quadratic forms from $K$ to $L$.
\begin{prop}\label{P:scalar-extension}
Consider a field $K$ and a field extension $L/K$.
For a symmetric bilinear space $(V, B)$ over $K$, there exists a unique symmetric bilinear form $B_L$ on $V_L = V \otimes_K L$ such that, for all $v, w \in V$ and $x, y \in L$, one has
$$ B_L(v \otimes x, w \otimes y) = B(v, w)xy.$$
Similarly, for a quadratic space $(V, q)$ over $K$, there exists a unique quadratic form $q_L$ on $V_L$ such that, for all $v \in V$ and $x \in L$, one has
$$q_L(v \otimes x) = x^2q(v) \quad \text{and} \quad \mf{b}_{q_L} = (\mf{b}_q)_L.$$
\end{prop}
\begin{proof}
By redoing the proof of \Cref{M-P:tensor-product-SBS}, using that $B$ is a $K$-bilinear map and also $L \times L \to L : (x, y) \mapsto xy$ is a $K$-bilinear map, one obtains that there exists a unique symmetric $K$-bilinear map $B_L : V_L \times V_L \to L$ such that $B_L(v \otimes x, w \otimes y) = B(v, w)xy$ for all $v, w \in V$ and $x, y \in L$.
One then readily verifies that this map is actually also $L$-bilinear.

For the second statement, let us first consider uniqueness.
If $q_L$ is a quadratic form on $V_L$ such that $q_L(v \otimes x) = x^2 q(v)$ for all $v \in V$ and $x \in L$, then clearly $\mf{b}_{q_L} = (\mf{b}_q)_L$.
But $q_L$ is completely determined by its values on elementary tensors and by $\mf{b}_{q_L}$.
This shows uniqueness.

If $\charac(K) \neq 2$, then the existence part of the statement follows from the fact that $q(v) = \frac{1}{2}\mf{b}_q(v, v)$ for all $v \in V$: one may just define $q_L(\alpha) = \frac{1}{2}(\mf{b}_q)_L(\alpha, \alpha)$ for $\alpha \in V_L$.
If $\charac(K) = 2$ then a more subtle argument is needed: one still has that there exists some bilinear (but not necessarily symmetric) form $B : V \times V \to K$ such that $q(v) = B(v, v)$ for all $v \in V$ (see \autocite[Section 7]{ElmanKarpenkoMerkurjev}) and one may then set $q_L(\alpha) = B_L(\alpha, \alpha)$ for $\alpha \in V_L$.
\end{proof}
\begin{defi}\label{D:scalar-extension}
For a symmetric bilinear space $(V, B)$ over $K$ and a field extension $L/K$ the symmetric bilinear space $(V, B)_L = (V_L, B_L)$ over $L$ constructed in \Cref{P:scalar-extension} is called the \emph{scalar extension of $(V, B)$ to $K$}.\index{scalar extension}

Similarly, for a quadratic space $(V, q)$, we define the \emph{scalar extension of $(V, q)$ to $L$} as the quadratic space $(V, q)_L = (V_L, q_L)$ constructed in \Cref{P:scalar-extension}.

For a quadratic space $(V, q)$ over $K$, we will say that it is \emph{isotropic over $L$} (respectively \emph{anisotropic, hyperbolic, multiplicative, a Pfister form, ... over $L$}) if $q_L$ is isotropic (respectively anisotropic, hyperbolic, multiplicative, ...).
\end{defi}
\begin{rem}
For a homogeneous degree $2$ polynomial $f \in K[X_1, \ldots, X_n]$ and a field extension $L/K$, we can consider $f$ as a polynomial over $L$.
We then have $(K^n, q_f)_L = (L^n, q_f)$.
\end{rem}
One verifies easily that for quadratic spaces $(V, q)$, $(V', q')$ one has that $(V, q) \cong (V', q')$ implies $(V_L, q_L) \cong (V'_L, q'_L)$, that $(q \perp q')_L \cong q_L \perp q'_L$, $(q \otimes q')_L \cong q_L \otimes q'_L$, and $(\mbb{H}_K)_L = \mbb{H}_L$. Furthermore, if $(V, q)$ is an $n$-fold Pfister form, then so is $(V_L, q_L)$.
Putting this together, we obtain the following:
\begin{prop}\label{P:restriction-homomorphism}
Assume $\charac(K) \neq 2$, let $L/K$ be a field extension.
The rule
$$ r_{L/K} : WK \to WL : [(V, q)] \mapsto [(V_L, q_L)] $$
gives a well-defined ring homomorphism.
For $n \in \nat$, we have $r_{L/K}(I^n K) \subseteq I^nL$.
\end{prop}
\begin{defi}\label{D:restriction-homomorphism}
Let $L/K$ be a field extension.
The map $r_{L/K}$ defined in \Cref{P:restriction-homomorphism} is called the \emph{restriction homomorphism}.\index{restriction homomorphism}
\end{defi}

\subsection{Exercises}
\begin{enumerate}
\item Let $p \in \mbb P$.
How many $2$-dimensional anisotropic quadratic forms exist over $\qq_p$, up to isometry?
And how many $3$-dimensional anisotropic quadratic forms, up to isometry?
\item Determine completely the set of all $p \in \mbb P$ for which $\langle 22, 42, 231, 345 \rangle_{\qq_p}$ is anisotropic.
\item Let $K$ be a field, $q$ be a non-singular $4$-dimensional quadratic form over $K$ of discriminant $d$.
Show that $q_{K[\sqrt{d}]}$ is similar to a $2$-fold Pfister form.
\item The proof of \Cref{P:classify-3-dimensional-qqp} relies on \Cref{M-P:symb-equal}, which in turn relies on the deep \Cref{M-T:Arason-Pfister}, which we have not proven in this course.
Can you restructure the arguments in this section to avoid using \Cref{M-P:symb-equal} or \Cref{M-T:Arason-Pfister}? (\textit{Hint:} First find a proof that $I^3 \qq_p = 0$ for all $p \in \mbb P$.)
\end{enumerate}

\end{document}